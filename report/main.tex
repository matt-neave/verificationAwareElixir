\documentclass[a4paper, twoside]{report}

%% Language and font encodings
\usepackage[english]{babel}
\usepackage[utf8x]{inputenc}
\usepackage[T1]{fontenc}

\usepackage{lmodern}

%% Sets page size and margins
\usepackage[a4paper,top=3cm,bottom=2cm,left=3cm,right=3cm,marginparwidth=1.75cm]{geometry}

%% Useful packages
\usepackage{amsmath}
\usepackage{mathtools}
\usepackage{amssymb}
\usepackage{graphicx}
\usepackage{listings}
\usepackage{float}
\usepackage{xcolor}
\usepackage{multicol}
\usepackage{backnaur}
\usepackage{geometry}
\usepackage{longtable}
\usepackage{array}
\usepackage{tcolorbox}
\usepackage{fancyvrb}
\usepackage{framed}
\usepackage{pifont}
\newcommand{\cmark}{\ding{51}}%
\newcommand{\xmark}{\ding{55}}%
\usepackage{booktabs}
\usepackage[colorinlistoftodos]{todonotes}
\usepackage[colorlinks=true, allcolors=blue]{hyperref}

% Code blocks
\definecolor{dkgreen}{RGB}{0,128,0} % Define the dkgreen color using RGB values
\lstset{
  frame=none,
  language=Promela,
  aboveskip=5mm,
  belowskip=5mm,
  showstringspaces=false,
  columns=flexible,
  xleftmargin=.3\textwidth,
  basicstyle={\small\ttfamily},
  numbers=none,
  numberstyle=\tiny\color{gray},
  keywordstyle=\color{blue},
  commentstyle=\color{dkgreen},
  stringstyle=\color{orange},
  breaklines=true,
  breakatwhitespace=true,
  tabsize=3,
  numbers=left, 
  captionpos=b, 
}

% Elixir code colouring
\definecolor{commentgreen}{RGB}{2,112,10}
\definecolor{eminence}{RGB}{108,48,130}
\definecolor{weborange}{RGB}{255,165,0}
\definecolor{frenchplum}{RGB}{129,20,83}

\lstdefinelanguage{elixir}{
    morekeywords={case,catch,def,defv,pre,post,do,else,false,%
        use,alias,receive,timeout,defmacro,defp,%
        for,if,import,defmodule,defprotocol,%
        nil,defmacrop,defoverridable,defimpl,%
        super,fn,raise,true,try,end,with,%
        unless, quote, unquote},
    otherkeywords={<-,->, |>, \%\{, \}, \{, \, (, )},
    sensitive=true,
    morecomment=[l]{\#},
    morecomment=[n]{/*}{*/},
    morecomment=[s][\color{purple}]{:}{\ },
    morestring=[s][\color{orange}]"",
    commentstyle=\color{commentgreen},
    keywordstyle=\color{eminence},
    stringstyle=\color{red},
	basicstyle=\ttfamily,
	breaklines,
	showstringspaces=false,
	frame=none,
    escapeinside={(*@}{@*)}
}
\lstdefinestyle{promela}{
    language=C,
    morekeywords={proctype, init, chan},
    keywordstyle=\color{blue},
    commentstyle=\color{gray},
    stringstyle=\color{orange},
    basicstyle=\ttfamily\footnotesize,
    showstringspaces=false,
    breaklines=true,
    frame=single,
    captionpos=b,
    tabsize=2
}
\lstdefinelanguage{dafny}{
    morekeywords={method, function, class, var, assert, ensures, requires, modifies, decreases, new, if, else, while, break, return, ghost, forall, exists, invariant, axiom, datatype},
    sensitive=true,
    morecomment=[l]{//},
    morecomment=[s]{/*}{*/},
    morestring=[b]",
    commentstyle=\color{green},
    keywordstyle=\color{blue},
    stringstyle=\color{red},
    basicstyle=\ttfamily,
    breaklines,
    showstringspaces=false,
    frame=none
}
\lstdefinelanguage{boogie}{
    morekeywords={implementation, procedure, assert, assume, havoc, call, return, var, function, modifies, ensures, requires, if, else, while, break, continue},
    sensitive=true,
    morecomment=[l]{//},
    morecomment=[s]{/*}{*/},
    morestring=[b]",
    commentstyle=\color{green},
    keywordstyle=\color{blue},
    stringstyle=\color{red},
    basicstyle=\ttfamily,
    breaklines,
    showstringspaces=false,
    frame=none
}
\lstdefinelanguage{none}{}

\title{Verlixir: Verification of Message-Passing Systems}
\author{Matthew Neave}
% Update supervisor and other title stuff in title/title.tex

\begin{document}
\input{title/title.tex}

\begin{abstract}
    Distributed algorithms are difficult to prove and reason about. With the rapid rise in cloud-based clusters, the need for developing robust distributed algorithms is seen as a critical requirement by service providers and has sparked new interest in developing tools for reasoning about distributed algorithms.
    \\ \\
    This project proposes and evaluates Verlixir, a verification-aware language for specifying and verifying message-passing systems. Verlixir supports three modes of operation. Simulation mode can be used to execute and observe the system in a controlled environment. Verification mode can verify the system against a set of properties. Finally, parameterization mode can be used to guarantee the system behaves across different configurations.
    \\ \\
    A qualitative evaluation of Verlixir demonstrates the capability to detect violations of properties in distributed algorithms, such as Paxos and Two-Phase Commit. Verlixir enhances Elixir programs with linear temporal logic properties, predicates and function contracts to specify the desired behaviour of systems. Property violations produce Elixir-friendly counterexamples that can be used to debug the system. Counterexamples show process interleaving and message-passing interleaving that results in a property violation.
    \\ \\
    Verlixir enables developers to define message-passing systems in terms of safety and liveness. Distributed algorithms can be prototyped and verified in a safe environment. Verlixir replaces the need for writing complex models in bespoke specification languages, which have to be maintained alongside implementation in modern programming languages. This is all achieved while maintaining the original program semantics, such that, after verification, the system can be run in production.
\end{abstract}
\renewcommand{\abstractname}{Acknowledgements}
\begin{abstract}
I am grateful to my supervisor, Dr. Naranker Dulay, for his guidance and support throughout this project. 
\\ \\
My sincerest gratitude goes to my family, for their continued support through my studies.
\\ \\
Finally, a word of thanks to my friends, for their encouragement and motivation.
\end{abstract}

\tableofcontents
% \listoffigures
% \lstlistoflistings
% \listoftables

% \chapter{Introduction}
With the rise of cloud-based clusters, developing robust distributed algorithms is becoming an increasingly difficult problem and the need for vigorous methodologies to verify the correctness of these algorithms has intensified. Distributed systems are interesting, as they provide performance and reliability benefits over centralised systems. They also provide scalability, modularity, and availability improvements \cite{cachin}.
\\ \\ 
Modern programming languages have been developed to support distributed algorithms that rely on message-passing as a means of communication. Common message-passing abstractions involve the use of channels (e.g. Go \cite{go}) or actors \cite{actor} (e.g. Erlang \cite{erlang}). At a high level, message-passing systems can be easier to reason about that a common alternative of shared memory. However, message-passing systems are often distributed across multiple nodes, which can introduce challenges in reasoning about the correctness of a system \cite{science_of_systems}.
\\ \\
Verification tools have been developed to support determining the correctness of systems. For example, first-order automated theorem provers such as Z3 \cite{z3} and formal specification languages like TLA+ \cite{tlaplus}. These tools allow systems to be modelled, and specifications to be defined that can then be used to prove properties over these systems. However, despite the power these tools provide, they often place a burden on developers to write and maintain models of systems alongside their actual implementation. This often leads to a paradigm shift away from system implementations that were designed in, for example, imperative programming languages such as C. Modern programming languages for example Dafny \cite{dafny} solve this issue by directly integrating Floyd-Hoare style logic verification alongside the implementation. This report aims to extend this notion to distributed, message-passing systems.
\\ \\
This report discusses the modelling of message-passing actor-based programs and the verification of their adherence to a specification, using Elixir as a target language to support the verification of real-world systems.
\section{Objectives}
This report introduces Verlixir, a verification-aware programming language for message-passing. Verlixir programs compile to byte-code to run on the BEAM VM, as well as guarantee system correctness under safety and liveness properties.
\\ \\
Much work has gone into verifying algorithms and programs such as various theorem provers and model checkers. While these tools were initially designed to allow developers to write specifications for how an algorithm should behave in bespoke specification language, more recently verification tools have been designed that can be directly applied to programs written in programming languages such as C \cite{c_to_promela}. A more recent advancement is support for verifying concurrent programs. However, much of this work has used global shared memory as an implementation for specifying process communication \cite{java_pathfinder}. This project sets out to accomplish the following objectives:
\begin{itemize}
    \item Integrate the verification of formal specifications into modern, message-based programming languages.
    \item Design a framework for simulating large-scale distributed systems.
    \item Check that the behaviour of a system is consistent across different configurations.
    \item Ensure verified systems can be run directly in production.
    \item Apply the aforementioned techniques and tooling to real-world systems implemented in Elixir.
\end{itemize}
The current research in the area of verifying modern programming languages presents many challenges for extending this notion to a message-passing system. State-of-the-art verification-aware languages such as Dafny avoid concurrent execution due to the challenges it can introduce to verification \cite{dafny_concurrency}. To verify a distributed system in this context, it is instead left to the user to model the system in a manner that it can be sequentially executed. Tools such as Gomela \cite{gomela} support verification of concurrent execution where communication is achieved across channels in Go. However, Go has a very different approach to the pure message-passing models we are interested in.
\section{Contributions}
To accomplish the objectives set out in this report, the following contributions have been made:
\begin{itemize}
    \item The primary contribution of this report is Verlixir. Verlixir is an analysis tool that parses Elixir programs and translates them into Promela \cite{promela} for model checking using the SPIN \cite{spin} model checker. Verlixir is capable of verifying multiple properties of highly concurrent Elixir programs and reporting back counterexamples in an Elixir-friendly format. Chapter \ref{chap:verlixir} provides an overview of what Verlixir is and how it can be used. We then provide a detailed explanation of the design and implementation of Verlixir in chapter \ref{chap:design}.
    \item The secondary contribution is LTLixir. LTLixir is a reduced subset of the standard Elixir language that has been extended to introduce linear temporal logic, propositional logic and contract design for specifying the behavior of a system alongside its implementation. Temporal properties specified in LTLixir can be verified using Verlixir. LTLixir is first introduced in section \ref{sec:ltlixir} and the design is discussed in section \ref{sec:specification_language}.
\end{itemize}
\chapter{Introduction}
With the rise of cloud-based clusters, developing robust distributed algorithms is becoming an increasingly difficult problem and the need for vigorous methodologies to verify the correctness of these algorithms has intensified. Distributed systems are interesting, as they provide performance and reliability benefits over centralised systems. They also provide scalability, modularity, and availability improvements \cite{cachin}.
\\ \\ 
Modern programming languages have been developed to support distributed algorithms that rely on message-passing as a means of communication. Common message-passing abstractions involve the use of channels (e.g. Go \cite{go}) or actors \cite{actor} (e.g. Erlang \cite{erlang}). At a high level, message-passing systems can be easier to reason about that a common alternative of shared memory. However, message-passing systems are often distributed across multiple nodes, which can introduce challenges in reasoning about the correctness of a system \cite{science_of_systems}.
\\ \\
Verification tools have been developed to support determining the correctness of systems. For example, first-order automated theorem provers such as Z3 \cite{z3} and formal specification languages like TLA+ \cite{tlaplus}. These tools allow systems to be modelled, and specifications to be defined that can then be used to prove properties over these systems. However, despite the power these tools provide, they often place a burden on developers to write and maintain models of systems alongside their actual implementation. This often leads to a paradigm shift away from system implementations that were designed in, for example, imperative programming languages such as C. Modern programming languages for example Dafny \cite{dafny} solve this issue by directly integrating Floyd-Hoare style logic verification alongside the implementation. This report aims to extend this notion to distributed, message-passing systems.
\\ \\
This report discusses the modelling of message-passing actor-based programs and the verification of their adherence to a specification, using Elixir as a target language to support the verification of real-world systems.
\section{Objectives}
This report introduces Verlixir, a verification-aware programming language for message-passing. Verlixir programs compile to byte-code to run on the BEAM VM, as well as guarantee system correctness under safety and liveness properties.
\\ \\
Much work has gone into verifying algorithms and programs such as various theorem provers and model checkers. While these tools were initially designed to allow developers to write specifications for how an algorithm should behave in bespoke specification language, more recently verification tools have been designed that can be directly applied to programs written in programming languages such as C \cite{c_to_promela}. A more recent advancement is support for verifying concurrent programs. However, much of this work has used global shared memory as an implementation for specifying process communication \cite{java_pathfinder}. This project sets out to accomplish the following objectives:
\begin{itemize}
    \item Integrate the verification of formal specifications into modern, message-based programming languages.
    \item Design a framework for simulating large-scale distributed systems.
    \item Check that the behaviour of a system is consistent across different configurations.
    \item Ensure verified systems can be run directly in production.
    \item Apply the aforementioned techniques and tooling to real-world systems implemented in Elixir.
\end{itemize}
The current research in the area of verifying modern programming languages presents many challenges for extending this notion to a message-passing system. State-of-the-art verification-aware languages such as Dafny avoid concurrent execution due to the challenges it can introduce to verification \cite{dafny_concurrency}. To verify a distributed system in this context, it is instead left to the user to model the system in a manner that it can be sequentially executed. Tools such as Gomela \cite{gomela} support verification of concurrent execution where communication is achieved across channels in Go. However, Go has a very different approach to the pure message-passing models we are interested in.
\section{Contributions}
To accomplish the objectives set out in this report, the following contributions have been made:
\begin{itemize}
    \item The primary contribution of this report is Verlixir. Verlixir is an analysis tool that parses Elixir programs and translates them into Promela \cite{promela} for model checking using the SPIN \cite{spin} model checker. Verlixir is capable of verifying multiple properties of highly concurrent Elixir programs and reporting back counterexamples in an Elixir-friendly format. Chapter \ref{chap:verlixir} provides an overview of what Verlixir is and how it can be used. We then provide a detailed explanation of the design and implementation of Verlixir in chapter \ref{chap:design}.
    \item The secondary contribution is LTLixir. LTLixir is a reduced subset of the standard Elixir language that has been extended to introduce linear temporal logic, propositional logic and contract design for specifying the behavior of a system alongside its implementation. Temporal properties specified in LTLixir can be verified using Verlixir. LTLixir is first introduced in section \ref{sec:ltlixir} and the design is discussed in section \ref{sec:specification_language}.
\end{itemize}
\chapter{Background}
\section[]{Communicating Sequential Processes}
\section[]{Model Checking}
Model checking is the process of determining if a finite-state machine (FSM) is correct under a provided specification. It typically involves enumerating all possible states of an FSM and ensuring the correctness of each state. In software development, model checkers are beneficial in providing guarantees for safety-critical systems as well as concurrent systems. Concurrent systems can often cause issues with uncommon instruction execution interleavings that are not easily identifiable until long into a runtime. For example, deadlocks can occur when instructions being run by two processes are dependent on one another making progress. A simple example of a deadlock that can occur is the following interleaving of instructions executed by two processes, $\tau_1$ and $\tau_2$. 
\[
\begin{aligned}
& \tau_1: \text{ acquire lock A} \\
& \tau_2: \text{ acquire lock B} \\
& \tau_1: \text{ acquire lock B} \\
& \tau_2: \text{ acquire lock A}
\end{aligned}
\]
This simple interleaving results in $\tau_1$ blocking until it can acquire lock B, and $\tau_2$ blocking until it can acquire lock A, hence the program is in a deadlock. Due to the nature of concurrent systems, we could run our program and never experience this interleaving of instructions from occurring, hence we could deem our program deadlock-free. By instead abstracting our program as a model, and verifying the correctness using a model checker, we could exhaustively check all possible states (interleavings of concurrent processes) and catch this deadlock. 
\par
Alongside determining progress can be made within a system, model checkers are also used to guarantee the correctness of a specification. To demonstrate, we model a very simple 24-hour clock, where at each time step, we progress time by an hour.
\[
\begin{aligned}
& \tau_1: \text{time} \leftarrow \text{time} + 1
\end{aligned}
\]
Unlike the previous example, this process can always make progress so will not result in a deadlock, however, it is not a correct implementation of a 24-hour clock. We would like our 24-hour clock to only represent times in the range 1 to 24. By introducing a specification alongside our model, we can use a model checker to determine if all the states of our program adhere to the specification. In this instance, we would just need to specify a bound over our time variable.
\[
\{ \text{time} \mid \text{time} \in \mathbb{N}, 1 \leq \text{time} \leq 24 \}
\]
This is a simple example of a specification, that we can write in a specification language and use in tandem with our model to check the correctness of using a model checker.
\subsection[]{A Comparison Of Model Checkers}
Many model checkers have been invented for this reason, each with different focuses and specification languages. This section will comment on some of the more common model checkers and discuss their functionalities. \\
\subsubsection*{\textbf{PAT}}
Process Analysis Toolkit (PAT) is a self-contained framework to support composing, simulating and reasoning of concurrent, real-time systems \cite{pat}. PAT is based on Tony Hoare's CSP and extends the language using its library called CSP\#. CSP\# is a superset language of the original CSP, hence all classical CSP models can be verified with PAT. PAT has shown to be capable of verifying classical concurrent algorithms such as the dining philosophers problem. Alongside its verification capabilities, the PAT toolkit can be used to simulate real-world scenarios over specifications. 
\par
PAT's ability to determine the correctness of classical process algebra means it is a strong, widely applicable model checker.

\subsubsection*{\textbf{BLAST}}
BLAST is an automatic verification tool for checking the temporal safety properties of C programs. Given a C program and a temporal safety property, BLAST either statically proves the program satisfies the property or provides an execution path that exhibits a violation of the property \cite{blast}.
\par
Where BLAST is more interesting than PAT is that it no longer relies on process algebra. The model checker is capable of being run directly on a subset of C programs, no intermediate modelling is required. As an end-user tool, this is more generally applicable than PAT, there is no burden on developers to think about how to model their systems with process algebra and instead can directly get safety guarantees from their programs. BLAST handles the translation of C programs to an abstract reachability tree (ART), a labeled tree that represents a portion of the reachable state space of the program. Using a context-free reachability algorithm on this representation of a C program means temporal properties can be checked without the end programmer being required to think about what the control-flow automata for the program will look like.
\par
BLAST falls short when model-checking large C programs. More importantly, it is unable to provide any guarantees on concurrent programs. A strong driving factor in why developers choose to design systems in Elixir is its concurrent capabilities. 

\subsubsection*{\textbf{PRISM}}
PRISM is a probabilistic model checker, a tool for formal modelling and analysis of systems that exhibit random behavior or probabilistic behavior \cite{prism}. It has been used to analyse systems implementing random distributed algorithms.

\subsubsection*{\textbf{TLC}}
In 1980, Leslie Lamport discovered the Temporal Logic of Action (TLA) \cite{tla}. TLA is a logic system for specifying and reasoning about concurrent systems. Both the systems and their properties are represented in the same logic so that the assertion that a system meets its specification can be expressed by a logical implication.
\par
TLA is capable of specifying complex systems but in a typically verbose manner. Leslie Lamport introduced TLA+ \cite{tlaplus}, combining mathematical ideas with concepts from programming languages to create a specification language that would allow mathematicians to write specifications in 20 lines as opposed to 20 pages.
\begin{figure}[h]
    \centering
    \begin{verbatim}
    ---------------------- MODULE HourClock ----------------------
    EXTENDS Naturals
    VARIABLE hr
    HCini == hr \in (1 .. 12)
    HCnxt == hr' = IF hr # 12 THEN hr + 1 ELSE 1
    HC == HCini /\ [][HCnxt]_hr
    --------------------------------------------------------------
    THEOREM HC => []HCini
    ==============================================================
    \end{verbatim}
    \caption{An example TLA+ Specification for an HourClock \cite{tlaplus}}
    \label{fig:hourclock_spec}
\end{figure}
\par
Furthering on from Leslie Lamport's discovery of these specification languages, Lamport created TLC \cite{tlc}, a model checker for the verification of TLA+ specifications. Similarly to BLAST, TLC builds a finite-state machine from the specification so the model checker can verify and debug invariance properties over it. TLC has been used to verify many large-scale, real-world systems specified in TLA+. Not only does it verify temporal properties of TLA+ specifications, but it can also model check PlusCal \cite{pluscal} algorithms. PlusCal is an algorithm language aimed to resemble that of pseudocode, but PlusCal algorithms can be automatically translated to TLA+ specifications to be reasoned about formally with TLC. We have already come across the concept of model-checking algorithms as opposed to specifications with BLAST, but instead of being strictly bound to the C programming language, PlusCal provides a more general framework agnostic of a choice of programming language allowing developers to separate reasoning about algorithms from their respective programs.

\subsubsection*{\textbf{SPIN}}
SPIN is an efficient verification system for models of distributed software systems. It has been used to detect design errors in applications ranging from high-level descriptions of distributed algorithms to detailed code \cite{spin}. Spin has a specification language, Promela, which the model checker uses to prove the correctness of asynchronous process interactions. Spin supports asynchronous process communication through channels, where processes can send and receive messages. Spin constructs labeled transition systems for respective processes from Promela specifications which it goes on to use for scheduling and to reason about properties of the model. Because many programming languages, such as GO \cite{go} rely on the creation of channels for asynchronous communication between processes, Promela becomes a natural solution to modelling these systems. 
\begin{lstlisting}[numbers=left, captionpos=b, caption={Example of a Promela specification that enqueues a message in a channel}]
mtype = { HELLO };
chan channel = [10] of { mtype };

init {
    channel ! HELLO;
}
\end{lstlisting}
\section[]{Elixir}
\subsection[]{Shared Memory and Message Passing}
\subsection[]{Quote and Unquote}
\subsection[]{Metaprogramming}
\section[]{Existing Work}
\subsection[]{Lean}
\subsection[]{C Wolf}
\subsection[]{Dafny}
\subsection[]{Promela}
\subsection[]{Gomela}
\section[]{Modelling Elixir Programs in Promela}
\subsection[]{Basic Deadlock}
\subsection[]{Dining Philosophers}
\subsection[]{Preconditions and Postconditions}


\chapter{Promela and Elixir} \label{chap:promela_elixir}
In chapter \ref{chap:verlixir}, we will introduce the Verlixir tool. Verlixir involves the parsing of Elixir programs, which are translated into a formal model. This model is written in Process Meta Language (Promela). This chapter will introduce both Promela and Elixir. We will go through the core concepts and syntactic elements that Verlixir relies upon to provide a verification-aware Elixir. 
\section{Promela} \label{sec:promela}
Promela is the verification modelling language used by the Spin model checker, to specify concurrent processes modelling distributed systems \cite{spin}. This section will discuss some of the core features that allow systems to be modelled and verified with Spin. This section aims to give an overview of the syntax and control of Promela, so any specifications in later sections or the code artifact can be read.
\subsection{Types and Variables}
Promela is statically typed. Variables can be declared once within the current scope and then re-assigned throughout. Variables can be declared locally within the context of a process, or in the global scope, where memory is shared. The types available in Promela, and assignment to variables of these types is similar to many imperative programming languages. Promela supports the types bit, bool, byte, pid, short, int and unsigned. Variable declaration and assignment then naturally follows.
\[
\begin{aligned}
\text{int a} = 2;
\end{aligned}
\]
Promela supports \textbf{arrays}. Arrays are typed and declared with a fixed size. Array bounds are constant, so the size cannot change. Only single dimensional arrays are supported. The syntax for declaring an array is as follows.
\[
\begin{aligned}
\text{int array[10];}
\end{aligned}
\]
We can also extend the basic types using \texttt{typedef}. This allows us to define records of multiple nested types. We use these records to build a Promela library, to support model checking Elixir programs, using \textbf{embedded C} code and Promela \textbf{inlines}. Embedded C code cannot be model checked, but Promela inlines can be. These features allow us to extend Promela, and avoid some of the limitations discussed in section \ref{sec:promela_limitations} 
\subsection{Control Flow}
Promela supports some basic control flow concepts. Firstly, the \texttt{skip} expression can be used with no effect when executed, other than possibly changing the control of an executing process. The selection construct \texttt{if} can be used to evaluate expressions and execute sequences based on the evaluation of these expressions. The syntax of an if statement is unique in comparison to a typical programming language.
\begin{lstlisting}[language=promela, xleftmargin=.3\linewidth]
if
    :: 1 + 1 < 3 ->
        printf("Condition 1...");
    :: else -> 
        printf("No conditions matched");
fi
\end{lstlisting}
In Promela, $else$ is a reserved keyword that can be used in any condition. An $else$ condition will negate all the previous conditions.
Repetition can be achieved either through the \texttt{do} construct, through the use of labels or with \texttt{for} loops. We will primarily focus on \texttt{do}, as it is the most suitable for our modelling needs.
\begin{lstlisting}[language=promela, xleftmargin=.3\linewidth]
do
    :: a < 10 -> 
        a = a + 1;
    :: a < 10 ->
        a = a + 2;
    :: else -> 
        break;
od
\end{lstlisting}
Unlike \texttt{if}, which selects sequentially, \texttt{do} will non-deterministically select a true branch to execute. This means for the above example, for a given execution, we cannot say how many iterations are performed. The \texttt{break} keyword is reserved for explicitly breaking out of the loop.
\\ \\
An important concept in contract specification is the \textbf{assert} keyword. An assertion is a logical statement that is expected to be true at a given point in the program. If an assertion is false, a violation is reported.
\subsection{Processes}
An imperative component of understanding the power of the Spin model checker is understanding how processes can run concurrently. Every Promela model requires an initial process that is spawned in the initial system state and determines the control of the program from the initial state. The \texttt{init} keyword is reserved for this purpose. Other processes can be defined using the \texttt{proctype} keyword and then spawned with \texttt{run}. Each process is assigned a process id (pid) which can be accessed within the context of a process using globally defined read-only variable \texttt{\_pid}. We can now define two processes, a process active in the initial state and a second process that is spawned.
\begin{lstlisting}[language=promela, xleftmargin=.3\linewidth, caption={Defining and spawning processes in Promela}., label={fig:promela_processes}]
    proctype SomeProcess(int a) {
        printf("Do something with %d\n", a);
    }
    
    init {
        int p1;
        p1 = run SomeProcess(10);

        printf("Init process spawned at %d\n", _pid);
        printf("Process 1 spawned at %d\n", p1);
    }
\end{lstlisting}
Processes run independently of one another, so a parent process terminating will not necessarily result in the termination of a child. Spin sets a limit of 255 concurrently executing processes. Multiple processes can be spawned in a single transition by using the \textbf{atomic} construct, which will ensure that no spawning process is scheduled, until all atomic processes have been scheduled. Similarly to atomicity, \texttt{d\_step} can be used to enforce multiple statements are treated as a single indivisible step. Unlike \texttt{atomic}, \texttt{d\_step} cannot block or jump.
\\ \\
Instead of \texttt{init}, we could have used an \textbf{active proctype}. Every \textbf{active proctype} is spawned in the initial state, allowing for more than one process to initially run.
\subsection{Channels}
The final concept to briefly discuss, is the asynchronous communication primitive, channels. Promela allows channels to be specified using the predefined data type \texttt{chan}. To correctly specify communication, we often need to allow messages of multiple types to be written to channels. For this reason, Promela introduces \texttt{mtype} that allows for the introduction of symbolic names for constant values.
\[
\text{mtype = \{ BROADCAST \};}
\]
Now, we can define a channel that expects a message to contain multiple fields and is bound to contain a maximum of 10 messages at any time.
\[
\text{chan global\_broadcast = [10] of \{ mtype, int \};}
\]
We now input messages to the channel using the (!) operator.
\[
\text{global\_broadcast ! BROADCAST, 1;}
\]
Similarly, we read messages from the channel in a first-in, first-out (FIFO) order.
\[
\begin{aligned}
& \text{int x;} \\
& \text{global\_broadcast ? BROADCAST, x;}
\end{aligned}
\]
Where the variable $x$ stores the resulting \texttt{int}, assuming the first message in the channel is of type \texttt{BROADCAST}. Sending and receiving from channels also supports an alternative flavour. The (!!) and (??) operators are used for sorted insertion and random selection. \textbf{Sorted insertion} (!!), will insert a message into the channel in a sorted order, based on the first field of the message. \textbf{Random receive} (??), is not random. As opposed to FIFO, it will select the first message in the channel that matches a given pattern. We call this first-in, first-fireable-out (FIFFO).
\subsection{Promela Example}
We will now provide a simple example of a Promela specification. The specification models Dijkstra's Semaphore. It consists of two labelled processes, and an initial process to coordinate the system. A shared channel is used, with a buffer size of 0, which means the channel is blocking (rendezvous). Listing \ref{lst:dijkstra_semaphore} shows the Promela specification.
\begin{lstlisting}[language=promela, xleftmargin=.3\linewidth, caption={Dijkstra's Semaphore in Promela}, label={lst:dijkstra_semaphore}]
#define p	0
#define v	1

chan sema = [0] of { bit };

proctype dijkstra() {	
    byte count = 1;

    do
    :: (count == 1) ->
        sema!p; count = 0
    :: (count == 0) ->
        sema?v; count = 1
    od	
}

proctype user() {	
    do
    :: sema?p;
        /* critical section */
    sema!v;
        /* non-critical section */
    od
}

init {	
    run dijkstra();
    run user();
    run user();
    run user()
}
\end{lstlisting}
The semaphore guarantees that only one of the user processes can enter its critical section at a time.
\subsection{Limitations} \label{sec:promela_limitations}
Promela is a powerful language for modelling concurrent systems, but it has limitations for capturing the full complexity of a real-world system. In general, hand-translations of a system into Promela can avoid some of these limitations with careful design. However, we are approaching this limits from an Elixir perspective, where features of Elixir may not be easily translated into Promela.
\begin{itemize}
    \item \textbf{Compute}: Promela is not designed to model complex computations. It does not support floating-point arithmetic, so we are limited to working with integers.
    \item \textbf{Memory}: Promela does not support dynamic memory allocation. This means we cannot model systems that require dynamic memory allocation, for example, linked lists.
    \item \textbf{Functions}: Promela does not support functions. It has no notion of a function call or return. By extension, Promela does not support recursion.
    \item \textbf{Randomness}: a Spin execution may not be deterministic, but it cannot model true randomness.
    \item \textbf{Probability}: there is no mechanism for modelling probabilistic behaviour, all correctness claims are checked unconditionally.
    \item \textbf{Time}: there is no notion of a system block, or related time properties in Promela. This means we cannot model sleeping threads.
\end{itemize}
In chapter \ref{chap:design}, we will discuss how we can overcome some of these limitations. In particular, how we overcome the lack of functions and dynamic memory, by introducing a Promela library, which handles these features.
\subsection{Summary}
This basic introduction to the syntax of the Promela modelling language, aims to make the reader familiar with the syntax involved in writing Promela specifications. It is not an exhaustive guide but should form a basis for understanding specifications present in a later section or the code artifact.

\section{Elixir}
Elixir \cite{elixir} is a functional programming language built on top of Erlang \cite{erlang} that runs on the BEAM virtual machine \cite{beam}. It is commonly used for building distributed, fault-tolerant applications because it supports concurrency, communication and distribution. Elixir actors are uniquely identified with a process identifier (pid) and associated with an unbounded mailbox. Each mailbox supports communication between actors; one actor can send a message to another actor's mailbox, which is then enqueued and can be received in a First-In-First-Firable-Out (FIFFO) ordering. FIFFO is similar to First-In-First-Out (FIFO) where elements are dequeued in the order they are enqueued. However, Elixir supports receiving messages with pattern-matching such that messages are received in a FIFO order concerning a certain pattern.
\\ \\
BEAM is a virtual machine that executes user programs in the Erlang Runtime System (ERTS). BEAM is a register machine where all instructions operate on named registers containing Erlang terms such as integers or tuples.
\\ \\
We have recently seen companies adopting Elixir in industry, in particular in domains such as telecoms and instant messaging. The Phoenix Framework \cite{phoenix} is a framework for building interactive web applications natively in Elixir, that can take advantage of Elixir's multi-processing and fault tolerance to build scalable web applications. The audio and video communication application Discord \cite{discord} uses Elixir to manage its 11 million concurrent users and the Financial Times \cite{ft} have begun migrating from Java to Elixir to enjoy the much smaller memory usage by comparison.
\\ \\
Elixir supports multi-processing in two key ways: nodes and processes. Each node is an instance of BEAM (a single operating system process), when an Elixir program is executed, a new instance of BEAM is instantiated for it to run on. In contrast, an Elixir process is not an operating system process. An Elixir process is lightweight in terms of memory and CPU usage (even in comparison to threads that many other programming languages favour). Elixir processes can run concurrently with one another and are completely isolated from one another. Elixir processes communicate via message passing.
\begin{lstlisting}[language=Elixir, xleftmargin=.2\linewidth, caption={An example of spawn/1 and spawn/4 in Elixir for spawning a new lightweight process and a new Elixir node}]
    # Spawn a new process
    spawn(fn -> 1 + 2 end)

    # Create a new BEAM instance
    Node.spawn(:"node1@localhost", MyModule, :start, [])
\end{lstlisting}

In Elixir, a receive statement is used to read messages in the mailbox. The receive block looks through the mailbox for a message that matches a given pattern. If no messages match the pattern, the process will block until one does.
\begin{lstlisting}[language=Elixir, xleftmargin=.4\linewidth, caption={An example of spawn/1 and spawn/4 in Elixir for spawning a new lightweight process and a new Elixir node}]
    # Example send in Elixir
    send self(), {:hello, "world"}

    # Example receive block in Elixir
    receive do
        {:hello, msg} -> IO.puts msg
    end
\end{lstlisting}
\subsection{Verifiable Feature Set} \label{sec:verifiable_feature_set}
In chapter \ref{chap:verlixir}, we will introduce Verlixir. Verlixir is designed to support a reduced set of core Elixir constructs. We will introduce these constructs here to give an overview of what the tool is capable of supporting.
\\ \\
Everything in Elixir is an expression. This means that every piece of code returns a value. For example, an $if$ statement will return a value dependent on the branch taken. This means that any expression can be matched on, using Elixir's match (=) operator. We can use pattern matching to match on the shape of an expression's evaluation. The set of expressions supported by Verlixir are:
\begin{itemize}
    \item \textbf{Values}: any value of a basic, primitive type such as integers and booleans. Elixir also has a concept of $atoms$. An atom is identified by a preceding colon (:), and is followed by letters, digits, `\_', `@' or a string.
    \item \textbf{Variables}: Elixir is dynamically, strongly typed. Variables are bound to using the match operator. Variable names start with lowercase letters. The `\_' character can be used to match an expression of any shape.
    \item \textbf{Data structures}: structures such as lists and tuples are treated as values. Lists are dynamic, whereas tuples are fixed in size. Lists are written as [1, 2, 3] and tuples are written as \{1, 2, 3\}. We can match on the shape of these data structures using pattern matching.
    \item \textbf{Pattern matching}: pattern matching can be used to match on the shape of an expression. In the context of a conditional guard, values can be used to evaluate the shape of an expression, where as variables can be used to bind to the value of an expression. For example, the pattern ${:ok, value}$, can be used to assign to the variable $value$ if the expression is a tuple shape, with an atom $ok$ as the first element.
    \item \textbf{Functions}: functions are defined using the $def$ keyword. Functions can group multiple sequentially executable expressions. Any function can be called or spawned as a new process or node.
    \item \textbf{Modules}: functions are grouped into modules.
    \item \textbf{Message passing}: Elixir's actor model supports message passing. Messages are sent and received between actors and their mailboxes.
    \item \textbf{Control flow structures}: Elixir supports many control flow expressions. For example, $if$, $case$, $unless$ and $for$. All of these introduce a new scope. They are expressions and can be matched on.
\end{itemize}
The feature set supported has been shown expressive enough to support real-world systems in chapter \ref{chap:eval}. We omit some features of Elixir. For example, data structures like sets and maps are not supported. We determined lists sufficient for our purposes. The supported feature set can be easily extended to support other data structures.
\subsection{Type Specifications}
Type specifications are imperative for the correctness of Verlixir specifications. Verlixir supports some basic types such as \texttt{integer()}, \texttt{boolean()}, \texttt{atom()} and \texttt{pid()}. Type specifications are not enforced by the Elixir compiler, but tools such as Dialyzer and Verlixir rely them.
\\ \\
In type specifications, message types are typed as \texttt{atom()}. The atom \texttt{:ok} is reserved to identify a \textbf{non-returning function}. In Elixir, all functions return a value, so in this context, a `non-returning function', is a function that's value is never matched. We briefly demonstrate the type specifications for two functions, the first, is a non-returning function with no arguments and the second function, takes two arguments and returns an integer.
\begin{lstlisting}[language=Elixir, xleftmargin=.3\linewidth, caption={Valid type specification examples}.]
    (*@\fbox{@spec bind\_server() :: :ok}@*)
    def bind_server do
        ...
    end

    (*@\fbox{@spec add(integer(), integer()) :: integer()}@*)
    def add a, b do
        ...
    end
\end{lstlisting}
Notice ($::$) marks the return type of the function. If these values are matched in the function body, they should not be matched to a different type.
\\ \\
Within a correct Verlixir specification, any message should also be typed. To ensure this, any instance of a message should begin with an atom which we will refer to as the \textbf{message type}. For example, \texttt{\{:bind\}} and \texttt{\{:calculate, 10, 20\}} are valid specification messages. The message, \texttt{\{false, 15\}}, would be ignored by Verlixir, as it does not begin with an atom. 
\subsection{Summary}
In this chapter, we learned about Elixir, the programming language built on top of Erlang and we explored some basic approaches to designing concurrent systems with it. The next section will explore how these core tools can be used in tandem, to provide developers guarantees over large-scale, distributed Elixir-based systems.
\chapter{Veriflixir}
Veriflixir is the main project contribution. The Veriflixir toolchain supports the simulation and verification of a set of Elixir programs. This set is named LTLixir and is detailed in section \ref{sec:ltlixir}. This chapter aims to inform the reader of the constructs defined in LTLixir and how Veriflixir can be used to reason about them. \ref{sec:ltlixir} introduces the LTLixir language and its constructs. \ref{sec:verifiable} provides an example of specifying a verifiable system and how Veriflixir can be used to detect violations of a specification. The subsequent subsections provide further details of more interesting features of LTLixir, such as specifying temporal properties.
\section{LTLixir} \label{sec:ltlixir}
LTLixir is the multi-purpose specification language that compiles to BEAM byte-code and is supported for verification by Veriflixir. Primarily, LTLixir is a subset of Elixir supporting both sequential and concurrent execution. This subset is expressive enough to well-known distributed algorithms such as basic paxos \cite{basic-paxos} and the alternating-bit protocol \cite{ab-protocol}. LTLixir extends Elixir with constructs for specifying temporal properties, specifically LTL properties (where LTLixir derives its name) as well as Floyd-Hoare style logic for specifying pre- and post-conditions. Specifications can be parameterized to identify violations of properties on specific configurations. 
\par

\section{Constructing a Verifiable Elixir Program} \label{sec:verifiable}
\subsection{Detecting a Deadlock} \label{sec:deadlock}
\subsection{Inspecting a Trace} 
\subsection{Linear Temporal Logic} 
\subsection{Hoare-style Logic} 
\chapter{Implementation}
\section{Verification-Aware Elixir Toolchain}
\section{Extending Elixir}
\subsection{Linear Temporal Logic}
\subsection{Hoare-style Logic}
\section{Modelling Elixir Programs}
\subsection{Sequential Execution}
\subsubsection{Selection}
\subsubsection{Functions}
\subsection{Concurrent Execution}
\subsection{Memory Model}
\subsubsection{Actors, Mailboxes and Message Passing}
\subsubsection{Dynamic Lists}
\subsection{The Elixir Standard Library}
\subsubsection{For Comprehension}
\subsubsection{Standard Modules}
\section{Simulation and Verification}
\subsection{Simulation}
\subsection{Verification}
\section{Summary, Limitations and Future Work}

\chapter{Evaluation}
In this chapter, we aim to evaluate the expressiveness of Verlixir's design. First, we perform a qualatitive analysis of modeling classical distributed algorithms such as basic paxos in section \ref{sec:paxos} and the alternating-bit protocol in section \ref{sec:ab}. Section \ref{sec:vs} will delve into a comparison against current state-of-the-art model checking techniques for modern-day programming languages. Finally, we will explore the performance of Verlixir in a growing sytem in section \ref{sec:perf}.
\section{Analysing Distributed Systems}
Verifying the correctness of real-world distributed systems is a major motivation for this project. Critical real-time systems (such as in air-traffic control or healthcare  \cite{airlines,healthcare}) should not fail and should rely on rigerous verification techniques to guarantee production code is correct.  
\subsection{Basic Paxos} \label{sec:paxos}
Paxos is an example of a distributed algorithm \cite{paxos_simple}. It is a consensus algorithm, where many processes are tasked to agree on a value. Processes may propose what this value should be, but only one value should be agreed upon. The safety requirements (SR)s for consensus are:
\begin{itemize}
    \item \textbf{SR1}: Only a value that has been proposed may be chosen.
    \item \textbf{SR2}: Only a single value is chosen.
    \item \textbf{SR3}: A process never learns that a value has been chosen unless it actually has.
\end{itemize}
The system's liveness requirement is that a proposed value is eventually chosen and if a value if chosen then a process can learn the chosen value.
\subsubsection{Informal Specification}
There are many flavors to the paxos algorithm. We will informally present a basic, one-shot paxos. We introduce three roles in the system: proposer, acceptor and learner. The paxos algorithm performs two steps: prepare and accept. A proposer will broadcast a prepare message to all the acceptors, who will respond with a promise. When the proposer has received a promise from a quorum $q$ of acceptors, it will broadcast an accept message. If more than q acceptors accept, then the value is chosen, and the learners are informed.
\par
To evaluate the expressiveness of Verlixir, we first must write the paxos specification in LTLixir. The specifications of proposer, acceptor and learner are similar to those presented in pseudocode by Marzullo, Mei and Meling \cite{paxos_pseudocode}. We now present the key differences in our Elixir specification to a traditional paxos design.
%  For the remainder of the specification, we use the notations $A, \alpha, P, \pi, L, \lambda$ to respectively represent either a set or individual acceptor, proposer or learner.
\par
All processes contain two functions, a start function to introduce relevant initial configuration and a main loop to process messages. Every acceptor initializes an accepted proposal, value and minimum proposal to $-1$ and then processes $prepare$ and $accept$ messages until receiving a $terminate$ message, signifying consensus has been reached. A termination clause is important to ensure the completion of a round of paxos. A proposer receives its configuration in the form of a $bind$ message, before executing its protocol. If during phase two, when asking acceptors to accept a value, a quorum of acceptors rejects the proposal, the proposer will inform the system it has reached consensus on value $0$. Traditionally, the proposer would retry with a higher proposal number, but we aim to avoid infinite paths so instead introduce this terminating condition. The learner awaits a $learn$ message from all proposers. We only ever consider a single learner and the learner is also responsible for spawning the proposers and acceptors, choosing their values and assigning proposal numbers for the single round of paxos. We finally setup the learner such that it spawns three acceptors and two proposers. The learner decides the values the proposers will propose, which for this example will be $31$ and $42$. Of course, in a different context, these values may come from other sources within a larger system, however, notional values are sufficient for our purposes.
\par
With our implementation complete, we introduce the three safety requirements established. To achieve this, we introduce a value $final\_value$ which the learner receives from proposers. This value is initialized to $0$ and set to the agreed value of consensus. Let's specify the temporal formula required to express our safety requirements. We first introduce four predicates into our specification (note the use of $0$ both represents a state where consensus is unreached, or a value received from a rejected proposer).
\[
\begin{array}{l}
\text{predicate p: final\_value} == 31 \\
\text{predicate q: final\_value} == 42 \\
\text{predicate r: final\_value} \neq 0 \\
\text{predicate s: final\_value} == 0 \ \lor \ \text{final\_value} == 31 \ \lor \ \text{final\_value} == 42 \\
\end{array}
\]
We can now use the predicates to simplify the formulation of the safety requirements.
\begin{itemize}
    \item \textbf{SR1}: $\lozenge r$
    \item \textbf{SR2}: $\square ( ( p \rightarrow \neg \lozenge q ) \land ( q \rightarrow \neg \lozenge p ) )$ 
    \item \textbf{SR3}: $\square s$
\end{itemize}
We now have a complete specification of the basic paxos algorithm in LTLixir. Note that SR1 could be considered a liveness requirement, this is a result of slight modifications on the original SRs to align with our specific implementation decisions. We can run Verlixir on the model to verify the safety requirements. When we run the verification mode, we see that no SRs are violated. This justifies that both the informal paxos specification we defined is correct regarding our SRs and that the implementation of the specification is also correct.
\begin{lstlisting}[language=bash, xleftmargin=.3\linewidth]
    Model ran successfully. 0 error(s) found.
    The verifier terminated with no errors.
\end{lstlisting}
This gives a good indication that the expressiveness of Verlixir is sufficient to model and verify distributed systems. However, we also should investigate how Verlixir can express errors for a more complex system such as paxos. 
\subsubsection{Counter-example one}
We introduce a bug into the proposer's protocol. The proposer will now wait for a majority of acceptors to accept the proposal and only be rejected if a majority of acceptors reject the proposal. This is a violation of the protocol, as we only need a single rejection (within the accepting quorum) for a proposer to retry with a higher proposal number. We can now run the verifier on the model again to see if the bug is detected. Verlixir reports a violation of SR2, which is expected. In particular, we are told there is a violation SR2 due to $( final\_value == 31 )$. We can infer that the learner was informed the chosen value is $42$, but a later proposer informed the learner the chosen value is $31$. Verlixir detects this bug, informing us that SR2 was violated and then produces its counterexample. Digesting this counterexample can take some time, as the interleaving of process communication that triggers this bug involves approximately 50 messages and 800 steps. The full message log is available in the appendix. We will provide a simpler interpretation to help reason that Verlixir has correctly identified the bug (derived from the message log) in figure \ref{fig:paxos_1}.
\begin{figure}[h]
    \centering
    \includegraphics[width=0.7\textwidth]{images/paxos_2.png}
    \caption{Violation of paxos specification due to proposer bug. Note the figure only shows the ordering of receive events. We see that although $p1$ forms a quorum of $accepted$ messages from $\{a1, a2\}$. Although one of these acceptors rejects the proposal (by sending a higher proposal number than $p1$ expected), the bug would require a majority of acceptors to have rejected the proposal, so $p1$ asks the learner to learn its value regardless.}
    \label{fig:paxos_1}
\end{figure}
\subsubsection{Counter-example two}
We now explore a second counter-example, again, the paxos specification and message log can be found in the appendix. This time, we introduce a bug into the proposer, such that if the proposer receives a $\{prepared, proposalNumber, value\}$ message from an acceptor with a higher proposal number, it propagates this proposal number forward. A correct paxos implementation should keep the same proposal number, but propagate the value forward. We again get a violation of SR2, where the mutual exclusion of values is violated. The violation is the same as counter-example one but caused by a different interleaving. 
\par
TODO INSERT DIAGRAM OF MESSAGE INTERLEVING AND WHY IT FAILED?
\par
\subsection{Consistent Hash Ring}
Consistent hashing is a distributed hashing technique designed to support dynamic loads of nodes in a system \cite{consistent_hash}. It has been used in large real-world systems to help scalability and load balancing \cite{dynamo}. Consistent hashing requires choosing a hash space and distributing both system nodes and system requests over the hash space. The hash-space is logically considered a ring due to the wrap-around semantics of the distribution applied over the hashing function.
\par
We will look at a simple version of a consistent hash ring involving a handler and a ring manager. The handler receives requests from the outside world and sends these to the ring manager to be distributed. The ring manager is responsible for taking these requests and determining which node should be responsible for handling them. The ring can dynamically grow and shrink in size depending on the load from handlers.
\par
To model the system, we are primarily concerned with one liveness property. Every incoming request should eventually be forwarded to the correct node in the ring. A more detailed implementation may involve the nodes communicating to determine the correct node for requests, identify faulty nodes and handle hand-off when nodes join or leave the ring. We will abstract this behaviour within our ring manager for now, and introduce some temporal properties to specify the system's correctness. Firstly, we will introduce some predicates to help simplify the LTL formulae.
\[
    \begin{aligned}
    & \forall i \in \{1..4\} \ \text{predicate p\textsubscript{i}: assigned\_node == node\textsubscript{i}} \\
    & \forall j \in \{1..3\} \ \text{predicate r\textsubscript{j}: next\_request == V[j]} \\
    & \text{where } V = \{1 \rightarrow 42, 2 \rightarrow 31, 3 \rightarrow 25\}
    \end{aligned}
\]
These predicates $p_i$ specify assignments of a value to a node in the ring and $r_j$ specify the next request to be processed by the ring manager. We can now introduce our liveness property, which we do so by breaking into components to capture specific details of the system.
\[
    \begin{aligned}
    & \phi_1: \square ( r1 \rightarrow \lozenge p1 ) \\
    & \phi_2: \square ( r2 \rightarrow \lozenge p3 ) \\
    & \phi_3: \square ( r3 \land n\_nodes == 3 \rightarrow \lozenge p1 ) \\
    & \phi_4: \square ( r3 \land n\_nodes == 4 \rightarrow \lozenge p4 ) \\
    \end{aligned}
\]
We use these properties to ensure the correctness of the system, by using an understanding of how the system hashes requests to enable verification of evolving behaviour. For example, we use the variable $n\_nodes$ to distinguish between different behaviour patterns depending on the loads of the system. In particular, $\phi_1$ and $\phi_2$ ensure that the ring manager assigns requests to the next sequential node in the ring. $\phi_3$ is responsible for ensuring the wrap-around semantics, when the hash value of a request is larger than the last node's range, it should be assigned to the first node. $\phi_4$ is responsible for ensuring that as the ring grows, the ring manager adjusts its assignment of requests so that the new node now receives its relevant load.
\par
We can attach these LTL properties to the handler model, $S$, to determine that our incoming requests are being handled as expected. We can run them with Verlixir, which determines there are no violations of the properties, and our hashing is performing as intended.
\subsubsection{Evolving the System Requirements}
Up to now, we have been strict in our liveness properties. In other flavors of the system, it may not want to concern ourselves with the exact node a request is assigned to, but rather that the request is assigned to a node. Our current implementation enforces a synchronisation between the handler and the ring manager. Let's introduce a bug into the system that breaks this synchronization. Currently, our handler will wait for ring resizing to complete before sending more requests. We will modify the ring manager to dynamically resize asynchronously to the handler requests. This introduces a violation of our liveness properties, as we can no longer guarantee that every request is assigned to a specific node. 
\par
When we run Verlixir on the updated model, $S'$, we see that $S' \not\models \phi_4$. The erroneous message log, alongside both specifications, can be found in the appendix. We will provide a simplified interpretation of the message log to help reason that Verlixir has correctly identified the bug in figure \ref{fig:dht}.
\par
\begin{figure}[h]
    \centering
    \includegraphics[width=0.8\textwidth]{images/dht.png}
    \caption{Violating and accepting consistent hash ring implementations. The violating model shows the handler sending lookup requests without awaiting confirmation of ring resize. This violates the liveness property $\phi_4$, which specifically requires the manager to assign $31$ to node $4$. The accepting implementation waits for confirmation of a resize before continuing with requests. Note that $n\_nodes$ is the number of nodes the handler believes to be in the ring, not the actual number.}
    \label{fig:dht}
\end{figure}
\par
In this instance, instead of considering this an error, we may instead want to refine the system requirements. To do this, we can introduce a new liveness property to specify a weaker system, where we only care about requests being distributed to nodes.
\[
\phi_5: \square (\text{sent\_request} \rightarrow \lozenge \text{assigned\_node})
\]
Verlixir reports that $S \models \phi_5$ and $S' \models \phi_5$.
\subsection{Alternating-bit Protocol} \label{sec:ab}
\subsubsection{Informal Specification}
\subsection{Two-Phase Commit} \label{sec:2pc}
\section{Summary}
\chapter{Conclusion}
In this paper, we aimed to provide modelling and verification techniques for message-passing systems. To evaluate these techniques, we targetted Elixir, the actor-based, concurrent programming language. As part of our research, we introduced Verlixir, a framework to support a verification-aware Elixir.
\par
Verlixir supports verification of in-line specifications, written using predicate and temporal logic. We demonstrated that Verlixir is capable of verifying a multitude of safety and liveness properties specified for real-world distributed algorithms. 
\section{Future Work}
Throughout the course of research, we have identified several areas for future work. We have indicated some of these areas previously, but we will discuss them in more detail here. An obvious extension would involve supporting a larger Elixir feature set, but we will not discuss this here.
\par
A large inspiration for this project was the verification-aware language, Dafny. The Dafny language sets out to achieve similar claims to Verlixir, but takes a different approach. Dafny fundamentally relies on theorem proving for verification, whereas Verlixir uses model checking. In particular, Dafny transpiles to Boogie, a verification-aware intermediate language. Boogie currently then uses Z3 for theorem proving. Theorem proving provides formal proofs of correctness, that gives stronger claims than the pre- and post-condition checks we assert in model checking. We believe an ideal solution to verify message-passing systems could involve a combination of both theorem proving and model checking. We could extend Verlixir to support theorem proving within Elixir processes, and then continue to use model checking for verifying inter-process communication. This would provide a stronger guarantee of correctness for the system as a whole.
\par
Throughout table \ref{table:vs}, we explored the differences between Verlixir and other verification-aware languages. A key insight from this table is the lack of capability for injecting faults into models. Many distributed algorithms have been designed with an approach such that given there are no more than, some $f$ faults, the system will still follow the specification. In order to truly scrutinise these algorithms, we need the capability to inject faults into the system. A simple extension to Verlixir could involve injecting random terminating faults into processes, and then observing the system's fault tolerance under these conditions.
\par
A second insight gathered from \ref{table:vs} is the lack of support for verifying computation tree logic (CTL) in modern programming languages. CTL is a temporal logic that allows us to reason about paths in a system. A simple example of where CTL could be useful is in verifying the Paxos algorithm. To ensure fairness between all proposers, we could specify and verify a CTL property that states: there exists a path where each proposers proposal is accepted. We believe that extending Verlixir to support CTL would provide a more comprehensive verification framework for message-passing systems.
\section{Ethical Considerations}
Throughout the report we make reference to critical distributed-systems used in the contexts of air traffic control and healthcare. While the tools and research discussed are designed to improve the reliability of these systems, relying on them in isolation is not sufficient. Testing distributed systems is hard, and it is important rigorous testing is performed by multiple parties using multiple techniques.
\par
That said, we believe Verlixir provides an advancement in the verification of distributed systems.
\section{Final Remarks}
The goal for this project was to design a tool that could verify real-world distributed algorithms which are used in research and industry. Throughout researching Verlixir, many notional examples were used to demonstrate the capabilities of the tool. Through the evaluation, we have shown these capabilities extend to real, well-known algorithms such as Paxos and Raft. In achieving this, we believe Verlixir has the potential to be a valuable tool for verifying distributed systems.
\par
Historically, model checking has required hand-translation of code into a model. Verlixir lowers the barrier to entry for verifying systems. If a programmer can write their implementation using the provided LTLixir set, the system verification comes for free. We believe moving towards a world where verification-aware languages become the standard will greatly improve the reliability of code. Instead of programmers thinking about how to implement a system, instead they can focus on what the system should do. A shift in mindset away from implementation specifics towards reasoning about the safety and liveness of a system means that we can describe our systems is a more unified way.
\par
With the recent rise in artificially intelligent copilots \cite{attention_is, copilot_asset,safety_ai}, the future of programming could move towards a more declarative style. We programmers could simply be burdened to write the system specification, and a large-language model could generate the implementation.
% \chapter{Project Plan}
This chapter will discuss what needs to be done for the project to be successful, the paths that can be taken, areas that can be explored as an extension and fall-back positions in the limit of time.
\par
\section{The Artifact}
The key aim of the project is to produce a code artifact that can verify the correctness of Elixir programs. "Verify" is being used as an umbrella term for three potential components of the artifact. In no particular order:
\begin{itemize}
    \item Determining if a program is deadlock-free.
    \item Allowing users to write specifications about functions that are inline and verifiable.
    \item Verifying liveness properties. 
\end{itemize}
It would be difficult/infeasible to design a tool that can do all three from scratch and similarly, I have yet to find a tool that can do all three. Hence, my current path forward is to use a combination of existing tools where necessary to achieve these verification feats. The most appropriate tools I have identified for each case respectively are the SPIN model-checker, the Boogie IVL for theorem-proving and the TLC model-checker.
\par
Given these three tools, the plan would be to develop a command-line tool, that takes an Elixir project as an input, parses the Elixir code, and extracts the underlying model from the project to create an internal representation that can be used to generate models in the three target output grammers.
\par
The main difficulty of this project will lie in the design of Elixir. Elixir focuses on the concurrent execution of programs, using the actor model for communication between sequential processes executing in parallel. None of the target output tools have support for message passing and Boogie does not support concurrency. That means the main challenge of the project will be designing a framework for modelling programming languages that use message-passing as a first-class solution to communication. If the framework is well designed, this may not be limited to Elixir, the intermediate representation could be a target grammar that any message-passing oriented language can be translated into (such as Rust). However, for the scope of this project, modelling Elixir will be sufficient.
\par
Once a model has been extracted from an Elixir program, the next step will be code generation for the relevant tools, which can then be ran to determine the correctness of the program.
\section{Timeline and Milestones}
This section is a brief overview of what needs to happen and what has already began.
\subsubsection{Manual Translation (started)}
The first step involved understanding what an Elixir program looks like in the target representations. I spent time taking some basic Elixir programs and translating them into Promela (the specification language used by SPIN) to gather an understanding of how they can begin to be translated. I modelled four programs, an entirely sequentially executed program, a program that introduces a deadlock, a program that introduces a livelock and a program based on a deadlocking dining philosophers algorithm. While doing this, I made notes of how various components can be modelled in Promela as well as what is difficult to model. A brief summary of some findings:
\begin{itemize}
    \item In the basic deadlock model, a deadlock was detected.
    \item In the dining philosophers model (a more complex example of an Elixir program) a deadlock was detected.
    \item In the basic livelock model, the model-checker ran forever and did not detect the livelock. In theory, Promela should be detecting livelocks so more investigation needs to be done into how models need to be bound to allow for the successful detection.
    \item The key primitives unique to the actor model were able to be modelled with fair success in Promela.
\end{itemize}
As well as translating Elixir programs to Promela models, I spent some important time translating quoted expressions (equivalent to ASTs) to models, as Elixir provides the functionality to easily access them.
\subsubsection{Parsing (started)}
Parsing Elixir programs so they can be stored in an intermediate representation is important. Because of the access to quoted expressions, there is no need to parse Elixir grammar directly, instead parsing quoted expressions is easier. They are guaranteed to be well-formed, and when parsed you are left with an AST by nature. Work has begun here using a parser combinator library (pest) in Rust. The Elixir developers do not provide exhaustive documentation of the grammar of quoted expressions, so instead they have to be derived from examples and trial-and-error.
\subsubsection{Model Extraction (started)}
The main challenge of the project is model extraction. It introduces many questions that need answering:
\begin{itemize}
    \item What does a sequential execution of statements look like?
    \item How can functions be represented to be modelled in specifications that don't support functions (Promela)?
    \item How can Elixir recursion be modelled in specifications that don't support recursion (Promela)?
    \item How can concurrency processes be modelled as a sequential process (Boogie)?
    \item ho
    w can message-passing be modelled, in particular when specifications don't support shared memory?
\end{itemize}
Unfortunately, the list goes on the more you look into the Elixir language. A starting point will be modelling sequential programs.
\subsubsection{Promela code-gen (begin in term 2)}
The first model checker I aim to target is Promela. Although it doesn't natively support function calls or definitions it does support concurrency, therefore I deem it an easier milestone to reach. Once code-gen for Promela is implemented, it should be possible to begin proving Elixir programs are deadlock-free, which is a massive step towards being a verification-aware language.
\subsubsection{Extending Elixir with Metaprogramming (begin in term 3)}
It will be difficult within the scope of the project to allow any Elixir program without modification to be verified. Using metaprogramming, steps can be taken to introduce an Elixir library that allows developers to write code that is easier to verify. For example, developers can introduce bounds to mailboxes and recursive calls that make model extracting a lot more direct. Anything unbounded won't be verifiable, so I want to put the burden on the developer writing specifications instead of the artifact to approximate bounds.
\par
Metaprogramming can also be used to introduce pre- and post-conditions to Elixir to extend the support for verification.
\subsubsection{Boogie code-gen (term 3 / future work)}
After extending the Elixir language to allow the introduction of pre- and post-conditions, code-gen for Boogie programs can take place that allows stronger verification support for Elixir. I deem this an important part of the project but it depends on a lot of prior work being finished, so it's hard to determine if it fits on the timeline.
\subsubsection{Liveness (future work)}
It is unlikely that liveness will be implemented as part of the verification toolkit unless it is discovered one of the existing downstream tools supports it. I believe code-gen for a new tool will be required (such as TLA+) to achieve this, which is likely to be infeasible in the scope of the project, but on my radar.
% \chapter{Evaluation}
In this chapter, we aim to evaluate the expressiveness of Verlixir's design. First, we perform a qualatitive analysis of modeling classical distributed algorithms such as basic paxos in section \ref{sec:paxos} and the alternating-bit protocol in section \ref{sec:ab}. Section \ref{sec:vs} will delve into a comparison against current state-of-the-art model checking techniques for modern-day programming languages. Finally, we will explore the performance of Verlixir in a growing sytem in section \ref{sec:perf}.
\section{Analysing Distributed Systems}
Verifying the correctness of real-world distributed systems is a major motivation for this project. Critical real-time systems (such as in air-traffic control or healthcare  \cite{airlines,healthcare}) should not fail and should rely on rigerous verification techniques to guarantee production code is correct.  
\subsection{Basic Paxos} \label{sec:paxos}
Paxos is an example of a distributed algorithm \cite{paxos_simple}. It is a consensus algorithm, where many processes are tasked to agree on a value. Processes may propose what this value should be, but only one value should be agreed upon. The safety requirements (SR)s for consensus are:
\begin{itemize}
    \item \textbf{SR1}: Only a value that has been proposed may be chosen.
    \item \textbf{SR2}: Only a single value is chosen.
    \item \textbf{SR3}: A process never learns that a value has been chosen unless it actually has.
\end{itemize}
The system's liveness requirement is that a proposed value is eventually chosen and if a value if chosen then a process can learn the chosen value.
\subsubsection{Informal Specification}
There are many flavors to the paxos algorithm. We will informally present a basic, one-shot paxos. We introduce three roles in the system: proposer, acceptor and learner. The paxos algorithm performs two steps: prepare and accept. A proposer will broadcast a prepare message to all the acceptors, who will respond with a promise. When the proposer has received a promise from a quorum $q$ of acceptors, it will broadcast an accept message. If more than q acceptors accept, then the value is chosen, and the learners are informed.
\par
To evaluate the expressiveness of Verlixir, we first must write the paxos specification in LTLixir. The specifications of proposer, acceptor and learner are similar to those presented in pseudocode by Marzullo, Mei and Meling \cite{paxos_pseudocode}. We now present the key differences in our Elixir specification to a traditional paxos design.
%  For the remainder of the specification, we use the notations $A, \alpha, P, \pi, L, \lambda$ to respectively represent either a set or individual acceptor, proposer or learner.
\par
All processes contain two functions, a start function to introduce relevant initial configuration and a main loop to process messages. Every acceptor initializes an accepted proposal, value and minimum proposal to $-1$ and then processes $prepare$ and $accept$ messages until receiving a $terminate$ message, signifying consensus has been reached. A termination clause is important to ensure the completion of a round of paxos. A proposer receives its configuration in the form of a $bind$ message, before executing its protocol. If during phase two, when asking acceptors to accept a value, a quorum of acceptors rejects the proposal, the proposer will inform the system it has reached consensus on value $0$. Traditionally, the proposer would retry with a higher proposal number, but we aim to avoid infinite paths so instead introduce this terminating condition. The learner awaits a $learn$ message from all proposers. We only ever consider a single learner and the learner is also responsible for spawning the proposers and acceptors, choosing their values and assigning proposal numbers for the single round of paxos. We finally setup the learner such that it spawns three acceptors and two proposers. The learner decides the values the proposers will propose, which for this example will be $31$ and $42$. Of course, in a different context, these values may come from other sources within a larger system, however, notional values are sufficient for our purposes.
\par
With our implementation complete, we introduce the three safety requirements established. To achieve this, we introduce a value $final\_value$ which the learner receives from proposers. This value is initialized to $0$ and set to the agreed value of consensus. Let's specify the temporal formula required to express our safety requirements. We first introduce four predicates into our specification (note the use of $0$ both represents a state where consensus is unreached, or a value received from a rejected proposer).
\[
\begin{array}{l}
\text{predicate p: final\_value} == 31 \\
\text{predicate q: final\_value} == 42 \\
\text{predicate r: final\_value} \neq 0 \\
\text{predicate s: final\_value} == 0 \ \lor \ \text{final\_value} == 31 \ \lor \ \text{final\_value} == 42 \\
\end{array}
\]
We can now use the predicates to simplify the formulation of the safety requirements.
\begin{itemize}
    \item \textbf{SR1}: $\lozenge r$
    \item \textbf{SR2}: $\square ( ( p \rightarrow \neg \lozenge q ) \land ( q \rightarrow \neg \lozenge p ) )$ 
    \item \textbf{SR3}: $\square s$
\end{itemize}
We now have a complete specification of the basic paxos algorithm in LTLixir. Note that SR1 could be considered a liveness requirement, this is a result of slight modifications on the original SRs to align with our specific implementation decisions. We can run Verlixir on the model to verify the safety requirements. When we run the verification mode, we see that no SRs are violated. This justifies that both the informal paxos specification we defined is correct regarding our SRs and that the implementation of the specification is also correct.
\begin{lstlisting}[language=bash, xleftmargin=.3\linewidth]
    Model ran successfully. 0 error(s) found.
    The verifier terminated with no errors.
\end{lstlisting}
This gives a good indication that the expressiveness of Verlixir is sufficient to model and verify distributed systems. However, we also should investigate how Verlixir can express errors for a more complex system such as paxos. 
\subsubsection{Counter-example one}
We introduce a bug into the proposer's protocol. The proposer will now wait for a majority of acceptors to accept the proposal and only be rejected if a majority of acceptors reject the proposal. This is a violation of the protocol, as we only need a single rejection (within the accepting quorum) for a proposer to retry with a higher proposal number. We can now run the verifier on the model again to see if the bug is detected. Verlixir reports a violation of SR2, which is expected. In particular, we are told there is a violation SR2 due to $( final\_value == 31 )$. We can infer that the learner was informed the chosen value is $42$, but a later proposer informed the learner the chosen value is $31$. Verlixir detects this bug, informing us that SR2 was violated and then produces its counterexample. Digesting this counterexample can take some time, as the interleaving of process communication that triggers this bug involves approximately 50 messages and 800 steps. The full message log is available in the appendix. We will provide a simpler interpretation to help reason that Verlixir has correctly identified the bug (derived from the message log) in figure \ref{fig:paxos_1}.
\begin{figure}[h]
    \centering
    \includegraphics[width=0.7\textwidth]{images/paxos_2.png}
    \caption{Violation of paxos specification due to proposer bug. Note the figure only shows the ordering of receive events. We see that although $p1$ forms a quorum of $accepted$ messages from $\{a1, a2\}$. Although one of these acceptors rejects the proposal (by sending a higher proposal number than $p1$ expected), the bug would require a majority of acceptors to have rejected the proposal, so $p1$ asks the learner to learn its value regardless.}
    \label{fig:paxos_1}
\end{figure}
\subsubsection{Counter-example two}
We now explore a second counter-example, again, the paxos specification and message log can be found in the appendix. This time, we introduce a bug into the proposer, such that if the proposer receives a $\{prepared, proposalNumber, value\}$ message from an acceptor with a higher proposal number, it propagates this proposal number forward. A correct paxos implementation should keep the same proposal number, but propagate the value forward. We again get a violation of SR2, where the mutual exclusion of values is violated. The violation is the same as counter-example one but caused by a different interleaving. 
\par
TODO INSERT DIAGRAM OF MESSAGE INTERLEVING AND WHY IT FAILED?
\par
\subsection{Consistent Hash Ring}
Consistent hashing is a distributed hashing technique designed to support dynamic loads of nodes in a system \cite{consistent_hash}. It has been used in large real-world systems to help scalability and load balancing \cite{dynamo}. Consistent hashing requires choosing a hash space and distributing both system nodes and system requests over the hash space. The hash-space is logically considered a ring due to the wrap-around semantics of the distribution applied over the hashing function.
\par
We will look at a simple version of a consistent hash ring involving a handler and a ring manager. The handler receives requests from the outside world and sends these to the ring manager to be distributed. The ring manager is responsible for taking these requests and determining which node should be responsible for handling them. The ring can dynamically grow and shrink in size depending on the load from handlers.
\par
To model the system, we are primarily concerned with one liveness property. Every incoming request should eventually be forwarded to the correct node in the ring. A more detailed implementation may involve the nodes communicating to determine the correct node for requests, identify faulty nodes and handle hand-off when nodes join or leave the ring. We will abstract this behaviour within our ring manager for now, and introduce some temporal properties to specify the system's correctness. Firstly, we will introduce some predicates to help simplify the LTL formulae.
\[
    \begin{aligned}
    & \forall i \in \{1..4\} \ \text{predicate p\textsubscript{i}: assigned\_node == node\textsubscript{i}} \\
    & \forall j \in \{1..3\} \ \text{predicate r\textsubscript{j}: next\_request == V[j]} \\
    & \text{where } V = \{1 \rightarrow 42, 2 \rightarrow 31, 3 \rightarrow 25\}
    \end{aligned}
\]
These predicates $p_i$ specify assignments of a value to a node in the ring and $r_j$ specify the next request to be processed by the ring manager. We can now introduce our liveness property, which we do so by breaking into components to capture specific details of the system.
\[
    \begin{aligned}
    & \phi_1: \square ( r1 \rightarrow \lozenge p1 ) \\
    & \phi_2: \square ( r2 \rightarrow \lozenge p3 ) \\
    & \phi_3: \square ( r3 \land n\_nodes == 3 \rightarrow \lozenge p1 ) \\
    & \phi_4: \square ( r3 \land n\_nodes == 4 \rightarrow \lozenge p4 ) \\
    \end{aligned}
\]
We use these properties to ensure the correctness of the system, by using an understanding of how the system hashes requests to enable verification of evolving behaviour. For example, we use the variable $n\_nodes$ to distinguish between different behaviour patterns depending on the loads of the system. In particular, $\phi_1$ and $\phi_2$ ensure that the ring manager assigns requests to the next sequential node in the ring. $\phi_3$ is responsible for ensuring the wrap-around semantics, when the hash value of a request is larger than the last node's range, it should be assigned to the first node. $\phi_4$ is responsible for ensuring that as the ring grows, the ring manager adjusts its assignment of requests so that the new node now receives its relevant load.
\par
We can attach these LTL properties to the handler model, $S$, to determine that our incoming requests are being handled as expected. We can run them with Verlixir, which determines there are no violations of the properties, and our hashing is performing as intended.
\subsubsection{Evolving the System Requirements}
Up to now, we have been strict in our liveness properties. In other flavors of the system, it may not want to concern ourselves with the exact node a request is assigned to, but rather that the request is assigned to a node. Our current implementation enforces a synchronisation between the handler and the ring manager. Let's introduce a bug into the system that breaks this synchronization. Currently, our handler will wait for ring resizing to complete before sending more requests. We will modify the ring manager to dynamically resize asynchronously to the handler requests. This introduces a violation of our liveness properties, as we can no longer guarantee that every request is assigned to a specific node. 
\par
When we run Verlixir on the updated model, $S'$, we see that $S' \not\models \phi_4$. The erroneous message log, alongside both specifications, can be found in the appendix. We will provide a simplified interpretation of the message log to help reason that Verlixir has correctly identified the bug in figure \ref{fig:dht}.
\par
\begin{figure}[h]
    \centering
    \includegraphics[width=0.8\textwidth]{images/dht.png}
    \caption{Violating and accepting consistent hash ring implementations. The violating model shows the handler sending lookup requests without awaiting confirmation of ring resize. This violates the liveness property $\phi_4$, which specifically requires the manager to assign $31$ to node $4$. The accepting implementation waits for confirmation of a resize before continuing with requests. Note that $n\_nodes$ is the number of nodes the handler believes to be in the ring, not the actual number.}
    \label{fig:dht}
\end{figure}
\par
In this instance, instead of considering this an error, we may instead want to refine the system requirements. To do this, we can introduce a new liveness property to specify a weaker system, where we only care about requests being distributed to nodes.
\[
\phi_5: \square (\text{sent\_request} \rightarrow \lozenge \text{assigned\_node})
\]
Verlixir reports that $S \models \phi_5$ and $S' \models \phi_5$.
\subsection{Alternating-bit Protocol} \label{sec:ab}
\subsubsection{Informal Specification}
\subsection{Two-Phase Commit} \label{sec:2pc}
\section{Summary}
\begin{thebibliography}{9}
\bibitem{csp}
Communicating Sequential Processes
Available from: \url{http://www.usingcsp.com/cspbook.pdf}.
\bibitem{pat}
Process Analysis Toolkit (PAT) 3.5 User
Manual
Available from: \url{https://pat.comp.nus.edu.sg/wp-source/resources/OnlineHelp/pdf/Help.pdf}.
\bibitem{blast}
The software model checker BLAST
Available from: \url{https://www.sosy-lab.org/research/pub/2007-STTT.The_Software_Model_Checker_BLAST.pdf}.
\bibitem{prism}
PRISM Model Checker
Available from: \url{https://www.prismmodelchecker.org/}.
\bibitem{tla}
The Temporal Logic of Actions
Available from: \url{https://lamport.azurewebsites.net/pubs/lamport-actions.pdf}.
\bibitem{tlaplus}
Specifying Concurrent Systems with TLA+
Available from: \url{https://lamport.azurewebsites.net/tla/xmxx01-06-27.pdf}.
\bibitem{tlc}
Model Checking TLA+ Specifications
Available from: \url{https://lamport.azurewebsites.net/pubs/yuanyu-model-checking.pdf}.
\bibitem{pluscal}
The PlusCal Algorithm Language
Available from: \url{https://lamport.azurewebsites.net/pubs/pluscal.pdf}.
\bibitem{spin}
The Model Checker SPIN
Available from: \url{https://spinroot.com/spin/Doc/ieee97.pdf}.
\bibitem{go}
Build simple, secure, scalable systems with Go
Available from: \url{https://go.dev/}.
\bibitem{elixir}
Elixir is a dynamic, functional language for building scalable and maintainable applications.
Available from: \url{https://elixir-lang.org/}.
\bibitem{beam}
A brief introduction to BEAM
Available from: \url{https://www.erlang.org/blog/a-brief-beam-primer/}.
\bibitem{erlang}
Practical functional programming for a parallel world
Available from: \url{https://www.erlang.org/}.
\bibitem{phoenix}
Phoenix Peace of mind from prototype to production
Available from: \url{https://phoenixframework.org/}.
\bibitem{discord}
Discord
Available from: \url{https://discord.com/}.
\bibitem{ft}
Financial Times
Available from: \url{https://www.ft.com/}.
\bibitem{shared_memory_verification}
Verification of Concurrent Programs in Dafny
Available from: \url{https://addi.ehu.es/bitstream/handle/10810/23803/Report.pdf?isAllowed=y&sequence=2}.
\bibitem{dafny}
The Dafny Programming and Verification Language
Available from: \url{dafny.org}
\bibitem{defmacro}
Elixir, Macros, Our First Macro
Available from: \url{https://hexdocs.pm/elixir/macros.html#our-first-macro}.
\end{thebibliography}
\appendix
\chapter{First Appendix}

\subsubsection{Paxos with a bug introduced to the proposer}
\begin{lstlisting}[language=Elixir, xleftmargin=.1\linewidth]
import VaeLib

defmodule Acceptor do

  @spec start_acceptor() :: :ok
  def start_acceptor do
    acceptedProposal = -1
    acceptedValue = -1
    minProposal = -1
    accept_handler(acceptedProposal, acceptedValue, minProposal)
  end

  @spec accept_handler(integer(), integer(), integer()) :: :ok
  def accept_handler(acceptedProposal, acceptedValue, minProposal) do
    receive do
      {:prepare, n, proposer} ->
        if n > minProposal do
          send proposer, {:promise, acceptedProposal, acceptedValue}
          accept_handler(acceptedProposal, acceptedValue, n)
        else
          send proposer, {:promise, acceptedProposal, acceptedValue}
          accept_handler(acceptedProposal, acceptedValue, minProposal)
        end
      {:accept, n, value, proposer} ->
        if n >= minProposal do
          send proposer, {:accepted, n}
          accept_handler(n, value, n)
        else
          send proposer, {:accepted, minProposal}
          accept_handler(acceptedProposal, acceptedValue, minProposal)
        end
      {:terminate} ->
        IO.puts("Terminating acceptor")
    end
  end
end

defmodule Proposer do
  @spec start_proposer() :: :ok
  def start_proposer do
    receive do
      {:bind, acceptors, proposal_n, value, maj, learner} -> proposer_handler(acceptors, proposal_n, value, maj, learner)
    end
  end

  @spec proposer_handler(list(), integer(), integer(), integer(), integer()) :: :ok
  def proposer_handler(acceptors, proposal_n, value, maj, learner) do
    for acceptor <- acceptors do
      send acceptor, {:prepare, proposal_n, self()}
    end

    receive_prepared(proposal_n, value, maj, 0, 0)
    {prepared_n, prepared_value} = receive do
      {:majority_prepared, n, v} -> {n, v}
    end

    for acceptor <- acceptors do
      send acceptor, {:accept, prepared_n, prepared_value, self()}
    end

    accepted_n = receive_accepted(maj, prepared_n, 0, 0)

    if accepted_n != -1 do
      # Value chosen
      send learner, {:learned, prepared_value}
    else
      # Value was rejected
      send learner, {:learned, 0}
    end
  end

  @spec receive_prepared(integer(), integer(), integer(), integer(), integer()) :: :ok
  def receive_prepared(proposal_n, value, maj, highest_seen_proposal, count) do
    if count >= maj do
      send self(), {:majority_prepared, proposal_n, value}
    else
      receive do
        {:promise, acceptedProposal, acceptedValue} ->
          if acceptedValue != -1 && acceptedProposal > highest_seen_proposal do
            receive_prepared(proposal_n, acceptedValue, maj, acceptedProposal, count + 1)
          else
            receive_prepared(proposal_n, value, maj, highest_seen_proposal, count + 1)
          end
      end
    end
  end

  @spec receive_accepted(integer(), integer(), integer(), integer()) :: integer()
  def receive_accepted(maj, prepared_n, rejections, count) do
    if count >= maj do
      if rejections >= maj do  # BUG IS HERE
        -1
      else
        prepared_n
      end
    else
      receive do
        {:accepted, n} ->
          if n > prepared_n do
            receive_accepted(maj, prepared_n, rejections + 1, count + 1)
          else
            receive_accepted(maj, prepared_n, rejections, count + 1)
          end
      end
    end
  end
end

defmodule Learner do

  @spec start() :: :ok
  @vae_init true
  def start do
    n_acceptors = 3
    quorum = 2
    n_proposers = 2
    vals = [42, 31]
    acceptors = for _ <- 1..n_acceptors do
      spawn(Acceptor, :start_acceptor, [])
    end

    for i <- 1..n_proposers do
      proposer = spawn(Proposer, :start_proposer, [])
      val_i = i - 1
      val = Enum.at(vals, val_i)
      send proposer, {:bind, acceptors, i, val, quorum, self()}
    end
    wait_learned(acceptors, n_proposers, 0)
  end

  @spec wait_learned(list(), integer(), integer()) :: :ok
  @ltl "[]((p->!<>q) && (q->!<>p))"
  @ltl "<>(r)"
  @ltl "[](s)"
  def wait_learned(acceptors, p_n, learned_n) do
    if p_n == learned_n do
      for acceptor <- acceptors do
        send acceptor, {:terminate}
      end
    else
      receive do
        {:learned, final_value} ->
          predicate p, final_value == 31
          predicate q, final_value == 42
          predicate r, final_value != 0
          predicate s, final_value == 0 || final_value == 31 || final_value == 42
          IO.puts("Learned final_value:")
          IO.puts(final_value)
      end
      wait_learned(acceptors, p_n, learned_n + 1)
    end
  end
end

\end{lstlisting}

\subsubsection{Paxos bug message log}
\begin{lstlisting}[xleftmargin=.01\linewidth, xrightmargin=0.01\linewidth, caption={Message passing caused by the proposer's protocol bug.}, label={lst:paxos_bug}]
    Never claim moves to line 6     [(1)]
138:    proc  7 (start_proposer:1) test_out.pml:314 Recv 7,BIND,0,0,2,0,0,0,0,0,1,0,0,0,0,0,42,0,0,0,0,0,2,0,0,0,0,0,0,0,0,0,0,0,0,0,0,0        <- queue 20 (__BIND)
154:    proc  8 (proposer_handler:1) test_out.pml:350 Send 1,PREPARE,0,0,1,0,0,0,0,0,7,0,0,0,0,0,0,0,0,0,0,0,0,0,0,0,0,0,0,0,0,0,0,0,0,0,0,0    -> queue 23 (__PREPARE)
158:    proc  8 (proposer_handler:1) test_out.pml:350 Send 3,PREPARE,0,0,1,0,0,0,0,0,7,0,0,0,0,0,0,0,0,0,0,0,0,0,0,0,0,0,0,0,0,0,0,0,0,0,0,0    -> queue 23 (__PREPARE)
162:    proc  8 (proposer_handler:1) test_out.pml:350 Send 5,PREPARE,0,0,1,0,0,0,0,0,7,0,0,0,0,0,0,0,0,0,0,0,0,0,0,0,0,0,0,0,0,0,0,0,0,0,0,0    -> queue 23 (__PREPARE)
197:    proc  6 (accept_handler:1) test_out.pml:262 Recv 5,PREPARE,0,0,1,0,0,0,0,0,7,0,0,0,0,0,0,0,0,0,0,0,0,0,0,0,0,0,0,0,0,0,0,0,0,0,0,0      <- queue 23 (__PREPARE)
203:    proc  6 (accept_handler:1) test_out.pml:272 Send 7,PROMISE,0,0,-1,0,0,0,0,0,-1,0,0,0,0,0,0,0,0,0,0,0,0,0,0,0,0,0,0,0,0,0,0,0,0,0,0,0    -> queue 26 (__PROMISE)
205:    proc  9 (receive_prepared:1) test_out.pml:425 Recv 7,PROMISE,0,0,-1,0,0,0,0,0,-1,0,0,0,0,0,0,0,0,0,0,0,0,0,0,0,0,0,0,0,0,0,0,0,0,0,0,0  <- queue 26 (__PROMISE)
219:    proc  2 (accept_handler:1) test_out.pml:262 Recv 1,PREPARE,0,0,1,0,0,0,0,0,7,0,0,0,0,0,0,0,0,0,0,0,0,0,0,0,0,0,0,0,0,0,0,0,0,0,0,0      <- queue 23 (__PREPARE)
262:    proc  4 (accept_handler:1) test_out.pml:262 Recv 3,PREPARE,0,0,1,0,0,0,0,0,7,0,0,0,0,0,0,0,0,0,0,0,0,0,0,0,0,0,0,0,0,0,0,0,0,0,0,0      <- queue 23 (__PREPARE)
278:    proc  0 (:init::1) test_out.pml:521 Send 10,BIND,0,0,3,0,0,0,0,0,2,0,0,0,0,0,31,0,0,0,0,0,2,0,0,0,0,0,0,0,0,0,0,0,0,0,0,0       -> queue 20 (__BIND)
292:    proc  4 (accept_handler:1) test_out.pml:272 Send 7,PROMISE,0,0,-1,0,0,0,0,0,-1,0,0,0,0,0,0,0,0,0,0,0,0,0,0,0,0,0,0,0,0,0,0,0,0,0,0,0    -> queue 26 (__PROMISE)
300:    proc 11 (receive_prepared:1) test_out.pml:425 Recv 7,PROMISE,0,0,-1,0,0,0,0,0,-1,0,0,0,0,0,0,0,0,0,0,0,0,0,0,0,0,0,0,0,0,0,0,0,0,0,0,0  <- queue 26 (__PROMISE)
344:    proc  2 (accept_handler:1) test_out.pml:272 Send 7,PROMISE,0,0,-1,0,0,0,0,0,-1,0,0,0,0,0,0,0,0,0,0,0,0,0,0,0,0,0,0,0,0,0,0,0,0,0,0,0    -> queue 26 (__PROMISE)
346:    proc 10 (start_proposer:1) test_out.pml:314 Recv 10,BIND,0,0,3,0,0,0,0,0,2,0,0,0,0,0,31,0,0,0,0,0,2,0,0,0,0,0,0,0,0,0,0,0,0,0,0,0       <- queue 20 (__BIND)
370:    proc 14 (receive_prepared:1) test_out.pml:421 Send 7,MAJORITY_PREPARED,0,0,1,0,0,0,0,0,42,0,0,0,0,0,0,0,0,0,0,0,0,0,0,0,0,0,0,0,0,0,0,0,0,0,0,0 -> queue 46 (__MAJORITY_PREPARED)
372:    proc  8 (proposer_handler:1) test_out.pml:363 Recv 7,MAJORITY_PREPARED,0,0,1,0,0,0,0,0,42,0,0,0,0,0,0,0,0,0,0,0,0,0,0,0,0,0,0,0,0,0,0,0,0,0,0,0 <- queue 46 (__MAJORITY_PREPARED)
382:    proc  8 (proposer_handler:1) test_out.pml:386 Send 1,ACCEPT,0,0,1,0,0,0,0,0,42,0,0,0,0,0,7,0,0,0,0,0,0,0,0,0,0,0,0,0,0,0,0,0,0,0,0,0    -> queue 47 (__ACCEPT)
386:    proc  8 (proposer_handler:1) test_out.pml:386 Send 3,ACCEPT,0,0,1,0,0,0,0,0,42,0,0,0,0,0,7,0,0,0,0,0,0,0,0,0,0,0,0,0,0,0,0,0,0,0,0,0    -> queue 47 (__ACCEPT)
390:    proc  8 (proposer_handler:1) test_out.pml:386 Send 5,ACCEPT,0,0,1,0,0,0,0,0,42,0,0,0,0,0,7,0,0,0,0,0,0,0,0,0,0,0,0,0,0,0,0,0,0,0,0,0    -> queue 47 (__ACCEPT)
421:    proc 15 (proposer_handler:1) test_out.pml:350 Send 1,PREPARE,0,0,2,0,0,0,0,0,10,0,0,0,0,0,0,0,0,0,0,0,0,0,0,0,0,0,0,0,0,0,0,0,0,0,0,0   -> queue 23 (__PREPARE)
425:    proc 15 (proposer_handler:1) test_out.pml:350 Send 3,PREPARE,0,0,2,0,0,0,0,0,10,0,0,0,0,0,0,0,0,0,0,0,0,0,0,0,0,0,0,0,0,0,0,0,0,0,0,0   -> queue 23 (__PREPARE)
429:    proc 15 (proposer_handler:1) test_out.pml:350 Send 5,PREPARE,0,0,2,0,0,0,0,0,10,0,0,0,0,0,0,0,0,0,0,0,0,0,0,0,0,0,0,0,0,0,0,0,0,0,0,0   -> queue 23 (__PREPARE)
452:    proc 16 (accept_handler:1) test_out.pml:262 Recv 1,PREPARE,0,0,2,0,0,0,0,0,10,0,0,0,0,0,0,0,0,0,0,0,0,0,0,0,0,0,0,0,0,0,0,0,0,0,0,0     <- queue 23 (__PREPARE)
458:    proc 16 (accept_handler:1) test_out.pml:272 Send 10,PROMISE,0,0,-1,0,0,0,0,0,-1,0,0,0,0,0,0,0,0,0,0,0,0,0,0,0,0,0,0,0,0,0,0,0,0,0,0,0   -> queue 26 (__PROMISE)
460:    proc 12 (accept_handler:1) test_out.pml:282 Recv 5,ACCEPT,0,0,1,0,0,0,0,0,42,0,0,0,0,0,7,0,0,0,0,0,0,0,0,0,0,0,0,0,0,0,0,0,0,0,0,0      <- queue 47 (__ACCEPT)
466:    proc 12 (accept_handler:1) test_out.pml:291 Send 7,ACCEPTED,0,0,1,0,0,0,0,0,0,0,0,0,0,0,0,0,0,0,0,0,0,0,0,0,0,0,0,0,0,0,0,0,0,0,0,0     -> queue 48 (__ACCEPTED)
474:    proc 17 (accept_handler:1) test_out.pml:282 Recv 1,ACCEPT,0,0,1,0,0,0,0,0,42,0,0,0,0,0,7,0,0,0,0,0,0,0,0,0,0,0,0,0,0,0,0,0,0,0,0,0      <- queue 47 (__ACCEPT)
492:    proc 17 (accept_handler:1) test_out.pml:296 Send 7,ACCEPTED,0,0,2,0,0,0,0,0,0,0,0,0,0,0,0,0,0,0,0,0,0,0,0,0,0,0,0,0,0,0,0,0,0,0,0,0     -> queue 48 (__ACCEPTED)
494:    proc 13 (accept_handler:1) test_out.pml:262 Recv 3,PREPARE,0,0,2,0,0,0,0,0,10,0,0,0,0,0,0,0,0,0,0,0,0,0,0,0,0,0,0,0,0,0,0,0,0,0,0,0     <- queue 23 (__PREPARE)
500:    proc 13 (accept_handler:1) test_out.pml:272 Send 10,PROMISE,0,0,-1,0,0,0,0,0,-1,0,0,0,0,0,0,0,0,0,0,0,0,0,0,0,0,0,0,0,0,0,0,0,0,0,0,0   -> queue 26 (__PROMISE)
512:    proc 18 (receive_accepted:1) test_out.pml:459 Recv 7,ACCEPTED,0,0,1,0,0,0,0,0,0,0,0,0,0,0,0,0,0,0,0,0,0,0,0,0,0,0,0,0,0,0,0,0,0,0,0,0   <- queue 48 (__ACCEPTED)
542:    proc 21 (accept_handler:1) test_out.pml:262 Recv 5,PREPARE,0,0,2,0,0,0,0,0,10,0,0,0,0,0,0,0,0,0,0,0,0,0,0,0,0,0,0,0,0,0,0,0,0,0,0,0     <- queue 23 (__PREPARE)
554:    proc 20 (receive_prepared:1) test_out.pml:425 Recv 10,PROMISE,0,0,-1,0,0,0,0,0,-1,0,0,0,0,0,0,0,0,0,0,0,0,0,0,0,0,0,0,0,0,0,0,0,0,0,0,0 <- queue 26 (__PROMISE)
560:    proc 23 (accept_handler:1) test_out.pml:282 Recv 3,ACCEPT,0,0,1,0,0,0,0,0,42,0,0,0,0,0,7,0,0,0,0,0,0,0,0,0,0,0,0,0,0,0,0,0,0,0,0,0      <- queue 47 (__ACCEPT)
578:    proc 23 (accept_handler:1) test_out.pml:296 Send 7,ACCEPTED,0,0,2,0,0,0,0,0,0,0,0,0,0,0,0,0,0,0,0,0,0,0,0,0,0,0,0,0,0,0,0,0,0,0,0,0     -> queue 48 (__ACCEPTED)
580:    proc 21 (accept_handler:1) test_out.pml:272 Send 10,PROMISE,0,0,1,0,0,0,0,0,42,0,0,0,0,0,0,0,0,0,0,0,0,0,0,0,0,0,0,0,0,0,0,0,0,0,0,0    -> queue 26 (__PROMISE)
582:    proc 24 (receive_prepared:1) test_out.pml:425 Recv 10,PROMISE,0,0,-1,0,0,0,0,0,-1,0,0,0,0,0,0,0,0,0,0,0,0,0,0,0,0,0,0,0,0,0,0,0,0,0,0,0 <- queue 26 (__PROMISE)
602:    proc 25 (receive_prepared:1) test_out.pml:421 Send 10,MAJORITY_PREPARED,0,0,2,0,0,0,0,0,31,0,0,0,0,0,0,0,0,0,0,0,0,0,0,0,0,0,0,0,0,0,0,0,0,0,0,0        -> queue 46 (__MAJORITY_PREPARED)
616:    proc 15 (proposer_handler:1) test_out.pml:363 Recv 10,MAJORITY_PREPARED,0,0,2,0,0,0,0,0,31,0,0,0,0,0,0,0,0,0,0,0,0,0,0,0,0,0,0,0,0,0,0,0,0,0,0,0        <- queue 46 (__MAJORITY_PREPARED)
626:    proc 15 (proposer_handler:1) test_out.pml:386 Send 1,ACCEPT,0,0,2,0,0,0,0,0,31,0,0,0,0,0,10,0,0,0,0,0,0,0,0,0,0,0,0,0,0,0,0,0,0,0,0,0   -> queue 47 (__ACCEPT)
630:    proc 15 (proposer_handler:1) test_out.pml:386 Send 3,ACCEPT,0,0,2,0,0,0,0,0,31,0,0,0,0,0,10,0,0,0,0,0,0,0,0,0,0,0,0,0,0,0,0,0,0,0,0,0   -> queue 47 (__ACCEPT)
634:    proc 15 (proposer_handler:1) test_out.pml:386 Send 5,ACCEPT,0,0,2,0,0,0,0,0,31,0,0,0,0,0,10,0,0,0,0,0,0,0,0,0,0,0,0,0,0,0,0,0,0,0,0,0   -> queue 47 (__ACCEPT)
659:    proc 27 (receive_accepted:1) test_out.pml:459 Recv 7,ACCEPTED,0,0,2,0,0,0,0,0,0,0,0,0,0,0,0,0,0,0,0,0,0,0,0,0,0,0,0,0,0,0,0,0,0,0,0,0   <- queue 48 (__ACCEPTED)
681:    proc 29 (accept_handler:1) test_out.pml:282 Recv 5,ACCEPT,0,0,2,0,0,0,0,0,31,0,0,0,0,0,10,0,0,0,0,0,0,0,0,0,0,0,0,0,0,0,0,0,0,0,0,0     <- queue 47 (__ACCEPT)
699:    proc 29 (accept_handler:1) test_out.pml:291 Send 10,ACCEPTED,0,0,2,0,0,0,0,0,0,0,0,0,0,0,0,0,0,0,0,0,0,0,0,0,0,0,0,0,0,0,0,0,0,0,0,0    -> queue 48 (__ACCEPTED)
701:    proc 28 (receive_accepted:1) test_out.pml:459 Recv 10,ACCEPTED,0,0,2,0,0,0,0,0,0,0,0,0,0,0,0,0,0,0,0,0,0,0,0,0,0,0,0,0,0,0,0,0,0,0,0,0  <- queue 48 (__ACCEPTED)
707:    proc 30 (receive_accepted:1) test_out.pml:454 Send 1    -> queue 78 (ret)
711:    proc 22 (accept_handler:1) test_out.pml:282 Recv 1,ACCEPT,0,0,2,0,0,0,0,0,31,0,0,0,0,0,10,0,0,0,0,0,0,0,0,0,0,0,0,0,0,0,0,0,0,0,0,0     <- queue 47 (__ACCEPT)
717:    proc 27 (receive_accepted:1) test_out.pml:466 Recv 1    <- queue 78 (ret1)
719:    proc 27 (receive_accepted:1) test_out.pml:467 Send 1    -> queue 54 (ret)
721:    proc 18 (receive_accepted:1) test_out.pml:471 Recv 1    <- queue 54 (ret2)
723:    proc 18 (receive_accepted:1) test_out.pml:472 Sent 1    -> queue 22 (ret)
724:    proc  8 (proposer_handler:1) test_out.pml:396 Recv 1    <- queue 22 (ret2)
740:    proc 22 (accept_handler:1) test_out.pml:291 Send 10,ACCEPTED,0,0,2,0,0,0,0,0,0,0,0,0,0,0,0,0,0,0,0,0,0,0,0,0,0,0,0,0,0,0,0,0,0,0,0,0    -> queue 48 (__ACCEPTED)
742:    proc 30 (receive_accepted:1) test_out.pml:459 Recv 10,ACCEPTED,0,0,2,0,0,0,0,0,0,0,0,0,0,0,0,0,0,0,0,0,0,0,0,0,0,0,0,0,0,0,0,0,0,0,0,0  <- queue 48 (__ACCEPTED)
748:    proc  8 (proposer_handler:1) test_out.pml:401 Send 0,LEARNED,0,0,42,0,0,0,0,0,0,0,0,0,0,0,0,0,0,0,0,0,0,0,0,0,0,0,0,0,0,0,0,0,0,0,0,0   -> queue 90 (__LEARNED)
762:    proc 31 (receive_accepted:1) test_out.pml:454 Send 2    -> queue 89 (ret)
766:    proc 30 (receive_accepted:1) test_out.pml:471 Recv 2    <- queue 89 (ret2)
768:    proc 30 (receive_accepted:1) test_out.pml:472 Send 2    -> queue 81 (ret)
772:    proc 28 (receive_accepted:1) test_out.pml:471 Recv 2    <- queue 81 (ret2)
774:    proc 28 (receive_accepted:1) test_out.pml:472 Sent 2    -> queue 41 (ret)
775:    proc 15 (proposer_handler:1) test_out.pml:396 Recv 2    <- queue 41 (ret2)
791:    proc 19 (wait_learned:1) test_out.pml:558 Recv 0,LEARNED,0,0,42,0,0,0,0,0,0,0,0,0,0,0,0,0,0,0,0,0,0,0,0,0,0,0,0,0,0,0,0,0,0,0,0,0       <- queue 90 (__LEARNED)
Never claim moves to line 5     [(!(!((final_value==42))))]
Never claim moves to line 16    [(1)]
807:    proc 26 (accept_handler:1) test_out.pml:282 Recv 3,ACCEPT,0,0,2,0,0,0,0,0,31,0,0,0,0,0,10,0,0,0,0,0,0,0,0,0,0,0,0,0,0,0,0,0,0,0,0,0     <- queue 47 (__ACCEPT)
813:    proc 26 (accept_handler:1) test_out.pml:291 Send 10,ACCEPTED,0,0,2,0,0,0,0,0,0,0,0,0,0,0,0,0,0,0,0,0,0,0,0,0,0,0,0,0,0,0,0,0,0,0,0,0    -> queue 48 (__ACCEPTED)
815:    proc 15 (proposer_handler:1) test_out.pml:401 Send 0,LEARNED,0,0,31,0,0,0,0,0,0,0,0,0,0,0,0,0,0,0,0,0,0,0,0,0,0,0,0,0,0,0,0,0,0,0,0,0   -> queue 90 (__LEARNED)
823:    proc 32 (wait_learned:1) test_out.pml:558 Recv 0,LEARNED,0,0,31,0,0,0,0,0,0,0,0,0,0,0,0,0,0,0,0,0,0,0,0,0,0,0,0,0,0,0,0,0,0,0,0,0       <- queue 90 (__LEARNED)
spin: _spin_nvr.tmp:15, Error: assertion violated
spin: text of failed assertion: assert(!((final_value==31)))
Never claim moves to line 15    [assert(!((final_value==31)))]
spin: trail ends after 826 steps
\end{lstlisting}

% \bibliographystyle{alpha}
% \bibliography{bibs/sample}

\end{document}