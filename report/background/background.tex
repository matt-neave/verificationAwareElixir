\chapter{Background}
This chapter aims to provide all the required background knowledge to understand the concepts discussed in the report. The contribution of the report involves extending a concurrent, message-passing language to be verification-aware. We define a verification-aware language as a language that strongly couples specifications and implementations. To understand how this can be done with a message-passing language, we must first form strong fundamentals in concurrency and existing verification techniques.
\par
We start by introducing process algebra that can be used to denote the behaviour of processes executing in parallel. We then talk about concurrency at a higher level in section \ref{sec:concurrency} by introducing memory models, safety, liveness and fairness. Section \ref{sec:model_checking} will introduce model checking and explore some of the existing state-of-the-art model checkers. Finally, we will discuss what a verification-aware language is and why they are important to modern system design in \ref{sec:existing_work}
\section{Communicating Sequential Processes} \label{csp_section}
Communicating Sequential Processes (CSP) was discovered by Tony Hoare, providing us with a mathematical notation for defining processes and interactive systems \cite{csp_paper}. CSP provides a framework for reasoning about the behaviour of concurrent systems which has influenced distributed algorithms \cite{distributed_algorithms_na_lynch}, model checking \cite{model_checking} and many other related research fields. This section will give a brief introduction to the CSP process algebra, which we will use to model some simple parallel processes.
\par
CSP defines processes and events. The alphabet of a process, $\alpha P$ is the set of all events. For example, the alphabet of a student process $S$ could consist of two events.
\[
\alpha S = \{study, sleep\}
\]
The process with the alphabet $A$ which never engages in the events of A is called $STOP_A$. $STOP$ can be considered a constant, and it is used to define a process that never engages in any available action. $STOP$ is used to describe behaviour of a broken object, such that the object terminates unsuccessfully. Similarly, $SKIP_A$ is defined as a process which does nothing but terminates successfully. We can now construct a sequence of events for a process that takes three actions and then breaks.
\[
(study \rightarrow sleep \rightarrow study \rightarrow STOP_{\alpha S})
\]
Similarly, a sequence of events for a process that takes an action and then terminates successfully.
\[
(study \rightarrow SKIP_{\alpha S})
\]
Using the same alphabet, we now define two simple processes modelling a student $L$ and a strict teacher $T$, who never accepts students sleeping.
\[
\begin{aligned}
& L = (study \rightarrow sleep \rightarrow L) \\
& T = (study \rightarrow study \rightarrow T) 
\end{aligned}
\]
Note that both processes are recursively defined, hence a valid \texttt{trace} for $L$ could be $\langle study, sleep, study, sleep \rangle$.
\par
We can now introduce the process algebra for concurrency, using the parallel composition operator (`||'). To help reason about concurrent processes, we also introduce a fixed-point constructor, $\mu X \bullet F(X)$, to define anonymous recursive processes. This now lets us denote a process that behaves like a system of composed processes, where both processes have the same algebra $\alpha S$.
\[
(T \,||\, L)    
\]
Using the definitions for $T$ and $L$, and the recursive process definition, we have.
\[
(T \,||\, L) = (study \rightarrow STOP)    
\]
This is the composition of both processes. As both begin with a \texttt{study} event, so does the composition. However, after \texttt{study}, each component process is prepared to take a different event. As these are different, the processes can not agree on what action to take next. The resulting \texttt{STOP} is known as a deadlock. Alternatively, had the student and teacher processes been defined with behaviour composed without deadlock, we could describe that behaviour with process algebra. In this example, the student and teacher have multiple actions that can be taken, denoted by the choice, $|$, notation.
\[
\begin{aligned}
& \alpha = \{learn, revise, exam\} \\
& L = (learn \rightarrow L \,|\, exam \rightarrow L \,|\, revise \rightarrow exam \rightarrow L) \\
& T = revise \rightarrow (exam \rightarrow T \,|\, learn \rightarrow T) \\
& \\ 
& (T \,||\, L) = \mu X \bullet (revise \rightarrow exam \rightarrow X)
\end{aligned}
\]
The composition can be described as a single process. Also, note the use of recursion with the fixed-point operator $\mu X \bullet F(X)$ \cite[p.74]{csp_paper}, where $X$ marks recursion under the composed process. Under this model, to compose two processes, we require simultaneous participation of the same event from both processes. Therefore, each event in the composed process can be attributed to events in both individual participants.
\par
We extend this notion to concurrency by considering separate alphabets for each process. Again, take our student $L$ and teacher $T$. If the alphabets of both processes differ, an event in the alphabet of $L$ that is not in the alphabet of $T$, is of no concern to $T$, thus becomes a valid event in the composition of the processes. Similarly to before, events that are in both alphabets can only be composed if they can be simultaneously taken by both processes individually.
\[
\begin{aligned}
& \alpha L = \{study, exam\} \\
& \alpha T = \{teach, exam\} \\
& L = study \rightarrow exam \\
& T = teach \rightarrow exam \\
& \\ 
& (T \,||\, L) = \mu X \bullet ((study \rightarrow teach | teach \rightarrow study) \rightarrow exam)
\end{aligned}
\]
\par
Finally, we will briefly look at how Hoare modelled communication between processes. Hoare designed communication over channels. A pair $c.v$ represents communication taking place over a channel $c$ and $v$ is the value of a message being passed. Hoare describes the set of all messages that a process $P$ can communicate on channel $c$ as $\{ v \,|\, c.v \in \alpha P \}.$
\[
\alpha c(P) = \{ v \,|\, c.v \in \alpha P \}
\]
Functions to extract the channel and message components from the pair $c.v$ are also defined.
\[
\begin{aligned}
& channel(c.v) = c \\
& message(c.v) = v
\end{aligned}
\]
With this understanding, we can finally model sending and receiving messages to channels. Given a process $P$ and a value $v \in \alpha c(P)$, a process can output $v$ on the channel $c$ using the $!$ operator (similar to sending a message to a channel in GO \cite{go}).
\[
(c!v \rightarrow P)
\]
Similarly, we can read messages from channels (receive a message) using the $?$ operator. A process can input any value $x$ on the channel $c$, and then behave under $P(x)$.
\[
(c?x \rightarrow P(x))
\]
That concludes a brief overview of the process algebra proposed by Tony Hoare. We saw how processes are constructed from a sequence of events; how processes can be composed and the special communication event $c.v$ which allows message-passing over channels. Tools have been built from similar syntax and concepts, for example, Promela \ref{sec:promela}, which models channels and messages with a similar approach.
\section{Concurrency} \label{sec:concurrency}
Concurrency introduces the ability for multiple components of a program to execute out-of-order. We saw in the previous section how multiple processes can be composed and treated as a single execution. For example, given two processes with disjoint alphabets, the parallel composition can result in any interleaving. This section aims to explore further in-depth the principles of concurrency, moving away from a mathematical representation and looking at higher-level concepts such as consistency models and temporal logic. 
\par
Concurrency is a core concept in classical distributed algorithms, such as the Paxos algorithm \cite{basic_paxos}, Raft \cite{raft} and Dining Philosophers \cite{dining_philosophers}. The capability for concurrency has grown with better hardware. Concurrency introduces the idea of consistency models \cite{art_of_multiprocessor_programming} to help reason about executions of multi-threaded systems. Lamport introduced sequential consistency \cite{sc} as a strong safety property for concurrent systems. Sequential consistency can be informally reasoned about by considering a single-core processor: if multiple threads are executed in parallel on a single-core processor, only one instruction can be executed at a time. This means that the result of any execution forms a total order, consistent with the order of operations on each individual process. For example, consider a new composition where a sequence of events has been executed by a teacher, \((teach \rightarrow teach \rightarrow)\), and a student is scheduled to execute next. Under the sequential consistency model the student must observe the same order of events as the teacher.
\par
Weaker memory models exist, which allow us to model systems that do not guarantee sequential consistency (for example, systems running on multi-core processors). Under these models, the instructions of a thread may be reordered (i.e. execute out-of-order) which introduces weak behaviours that we would not observe under sequential consistency. For example, total store ordering is a weaker memory model, that allows the reordering of write-read operations on different memory locations within a single thread. For the examples discussed in this report, we will assume a sequentially consistent model.
\subsection{Temporal Logic}
First-order logic, or predicate logic uses quantifiers to reason about the truth of statements. For example, the statement $(\forall x \in \mathbb{N} \,.\, x > 0)$ is true for all natural numbers $\mathbb{N}$, and uses the universal quantifier $\forall$, to quantify over all x. We can also use the existential quantifier $\exists$, to reason about the existence of an element in a set. First-order logic is a powerful tool for reasoning about the truth of statements, but it cannot reason about time and change. We introduce modal logic for this purpose.
\begin{bnf*}
    \bnfprod{A}
      {p \bnfor \top \bnfor \neg \bnfpn{A} \bnfor \bnfpn{A} \land \bnfpn{A} \bnfor \square \bnfpn{A}}\\
\end{bnf*}
Where A is a modal formula, $p$ is an atomic proposition, $\top$ represents `truth' and $\square A$ reads box A. Using rules from first-order logic we can introduce disjunct, implication and if-and-only-if. We can also introduce the second modal operator, $\lozenge A$ which reads diamond A.
\[
\lozenge \langle A \rangle \models \neg \square \neg \langle A \rangle
\]
Depending on the circumstances that box and diamond are applied, they have different readings. For example, in temporal logic, $\square A$ can read as `always A' and $\lozenge A$ can read as `sometimes A', informally, it can be useful to think of $\square$ similarly to $\forall$ and $\lozenge$ to $\exists$.
\par
Saul Kripke introduced Kripke semantics \cite{kripke} for reasoning about temporal logic. For example, consider modelling a system based on our student and teacher processes. We let $M$ be the model of the system and $s$ represent a singular state the system can be in. Typically, $s$ is the initial state of the system. If we are given a temporal formula $A$, we can now define the syntax for the truth of $A$ in state $s$ of the model $M$.
\[
(M, s) \models A
\]
To reason formally about what it means for $A$ to hold in state $s$, Kripke provided formal definitions for the base and inductive definitions of $A$.
\par
We finally extend this understanding of temporal logic to linear temporal logic (LTL), or sometimes written as linear-time temporal logic. LTL allows us to reason about the time and change of a model. Before we provide some examples, we introduce a final temporal operator, $U$, which reads until. The formula $\phi U \psi$ defines the truth of $\phi$ until $\psi$ holds. We call this `strong until' as there must exist a state where $\psi$ becomes true. `Weak until' $W$, can also be defined, which loosens the restrictions such that $\phi$ could hold for the entire execution.
\par
We can now explore a few basic examples of LTL formulas as well as provide some intuition behind them. We define a set of atomic propositions, $AP = \{study, sleep, tired, exam\}$.
\begin{multicols}{2}
    \[
    \begin{aligned}
    &\square \ \text{sleep} \\
    &\text{tired} \, U \, \text{sleep} \\
    &\square (\text{study} \Rightarrow \text{tired}) \\
    &\square \ \text{study} \Rightarrow \lozenge \ \text{exam}
    \end{aligned}
    \]
    \vline
    \[
    \begin{aligned}
    \text{Always sleeping} \\
    \text{Tired until sleeping} \\
    \text{Studying implies always tired} \\
    \text{Always studying implies eventually an exam}
    \end{aligned}
    \]
    \end{multicols}
\par
Alongside LTL, other forms of temporal logic exist, such as Computation Tree Logic (CTL) \cite{temporal_and_modal_logic} and Alternating-time Temporal Logic (ATL) \cite{atl}. CTL introduces path quantifiers to reason about specific traces through a model, and ATL introduces the idea of agents, where agents can work in coalitions to achieve a goal in the system. Temporal logic is an important concept in the model checking of systems \cite{principles_of_model_checking}, see chapter \ref{sec:model_checking}. 
\subsection{Safety and Liveness}
Safety and liveness are properties that can be specified about systems. A safety property can be intuitively thought of as a property such that nothing bad happens, and a liveness property is where something good will happen. For example, something bad could be a deadlock in a system, and something good could be that the system will eventually reach a consensus. We define safety informally as: given a finite execution $E$ and a state $s$ such that $s$ is the final state in the execution, we can say that a safety property $P$ holds if $P$ is true in $s$ and all previous states in $E$. If $s$ violates $P$, then $E$ violates $P$. Unlike with safety, we cannot determine the truth of a liveness property at $s$, we must instead inspect an infinite execution $E'$. We express these properties as temporal formulas. For example, we can specify a simple liveness property to ensure our student will always study again.
\[
\square \lozenge \text{study}
\]
To help understand why this is a liveness property, consider two states, $s_1$ and $s_2$. Take the assignment of $study$ to be $\{s_1\}$ i.e., $study$ is true only in $s_1$. Regardless of if we want to reason about the truth of the formula at $s_1$ or $s_2$, we cannot, as the box operator requires the formula to hold in all states. Hence, we would have to inspect the current state, as well as an infinite future execution from the state, to determine the truth of the formula. Because no single state exists where we can evaluate the truth of the formula, we can convince ourselves it is a liveness property.
\par
We use the same assignment of study to reason about a new property.
\[
\square \ \text{study}
\]
We can understand intuitively why this property is a safety property by considering $s_2$. As $s_2$ is not in the assignment of study (i.e. $study$ is false in $s_2$), any execution that passes through $s_2$ will violate the property. As we can determine the truth of the property with a finite execution, we can deem the property a safety property.
\par
By using both safety and liveness properties, we can define a `correct' system, through the evaluation of these temporal formulae.
\subsection{Fairness}
Fairness introduces more properties that can be defined using temporal formulae. Fairness properties do not target the specification of the system in the same way that other properties we have looked at do. Instead, fairness properties are constraints on the scheduling of the system. They aim to fairly select which process to execute next. Without fairness, a system could favour the scheduling of process A while never progressing with process B. We will discuss two flavours of fairness, weak fairness and strong fairness. Properties that hold under weak fairness also hold under strong fairness, hence strong implies weak. To define the fairness properties, we must first define what it means for an event to be enabled. An event (or action) $A$ of a process algebra is enabled if it can be executed in the current state. We will use the notation $A_E$ to denote an enabled event, for example, $study_E$ denotes the study event is executable in the current state. We now define weak fairness (WF) and strong fairness (SF).
\[
\begin{aligned}
&WF \, A \equiv \lozenge \square A_E \Rightarrow \square \lozenge A \\
&SF \, A \equiv \square \lozenge A_E \Rightarrow \square \lozenge A \\
\end{aligned}
\]
We can informally define weak fairness as: if an event is continuously enabled, it is executed infinitely often. Similarly, strong fairness reads: if an event is repeatedly enabled, it is executed infinitely often.
\section{Model Checking} \label{sec:model_checking}
Model checking is the process of determining if a finite-state machine (FSM) is correct under a provided specification. It typically involves enumerating all possible states of an FSM and ensuring the correctness of each state. For example, given a model M and a property $\varphi$, if no state of M violates $\varphi$, then we can say M satisfies $\varphi$. In software development, model checkers are beneficial in providing guarantees for safety-critical systems as well as concurrent systems. Concurrent systems can often cause issues with uncommon instruction execution interleaving that are not easily identifiable until long into a runtime. For example, deadlocks can occur when instructions being run by two processes are dependent on one another making progress. A simple example of a deadlock that can occur, is the following interleaving of instructions executed by two processes, $\tau_1$ and $\tau_2$. 
\begin{multicols}{2}
    \[
    \begin{aligned}
    & \tau_1: \text{ acquire lock A} \\
    & \tau_1: \text{ acquire lock B} \\
    & \tau_1: \text{ release locks}
    \end{aligned}
    \]
    \vline
    \[
    \begin{aligned}
    & \tau_2: \text{ acquire lock B} \\
    & \tau_2: \text{ acquire lock A} \\
    & \tau_2: \text{ release locks}
    \end{aligned}
    \]
    \end{multicols}
An interleaving such as ($\tau_1$, $\tau_2$, $\tau_1$, $\tau_2$, \dots) results in $\tau_1$ blocking until it can acquire lock B, and $\tau_2$ blocking until it can acquire lock A, hence the program is in a deadlock. Due to the nature of concurrent systems, we could run our program and never experience this interleaving of instructions from occurring, hence we could deem our program deadlock-free. Instead, by abstracting our program as a model, and verifying the correctness using a model checker, we could exhaustively check all possible states (interleaving of concurrent processes) and catch this deadlock. 
\par
Alongside determining progress can be made within a system, model checkers are also used to guarantee the correctness of a specification. To demonstrate, we model a very simple 24-hour clock, where at each time step, we progress time by an hour.
\[
\begin{aligned}
& \tau_1: \text{time} \leftarrow \text{time} + 1
\end{aligned}
\]
Unlike the previous example, this process can always make progress so will not result in a deadlock, however, it is not a correct implementation of a 24-hour clock. We would like our 24-hour clock to only represent times in the range 1 to 24. By introducing a specification alongside our model, we can use a model checker to determine if all the states of our program adhere to the specification. In this instance, we would just need to specify a bound over our time variable.
\[
\{ \text{time} \mid \text{time} \in \mathbb{N}, 1 \leq \text{time} \leq 24 \}
\]
This is a simple example of a specification, that we can write in a specification language and use in tandem with our model to check the correctness of using a model checker.
\subsection{A Comparison Of Model Checkers}
Many model checkers have been invented for this reason, each with different focuses and specification languages. This section will comment on some of the more common model checkers and discuss their functionalities. We first provide an overview of the capabilities and limitations of many model checkers, before providing a more in-depth look into model checkers best aligned with the goals of this report.
\par
\subsubsection{Overview}

\begin{table}[ht]
    \centering
    \begin{tabular}{|>{\raggedright\arraybackslash}p{3cm}|
        >{\centering\arraybackslash}p{2cm}|
        >{\centering\arraybackslash}p{2cm}|
        >{\centering\arraybackslash}p{3cm}|
        >{\centering\arraybackslash}p{3cm}|}
        \hline
        \textbf{Model Checker} & \textbf{LTL Support} & \textbf{CTL Support} & \textbf{Probabilistic} & \textbf{Concurrency Support} \\
        \hline
        PAT & Yes & No & Yes & Yes \\
        BLAST & No & No & No & Limited \\
        SPIN & Yes & No & No & Yes \\
        TLC & Yes & No & No & Yes \\
        PRISM & Yes & Yes & Yes & Yes \\
        NuSMV & Yes & Yes & No & Yes \\
        UPPAAL & No & No & Yes & Yes \\
        \hline
    \end{tabular}
    \caption{Comparison of Model Checkers}
    \end{table}
\subsubsection*{\textbf{PAT}}
Process Analysis Toolkit (PAT) is a self-contained framework to support composing, simulating and reasoning of concurrent, real-time systems \cite{pat}. PAT is based on Tony Hoare's CSP and extends the language using its library called CSP\#. CSP\# is a superset language of the original CSP, hence all classical CSP models can be verified with PAT. PAT has shown to be capable of verifying classical concurrent algorithms such as the dining philosophers problem. Alongside its verification capabilities, the PAT toolkit can be used to simulate real-world scenarios over specifications. 
\par
PAT's ability to determine the correctness of classical process algebra means it is a strong, widely applicable model checker.

\subsubsection*{\textbf{BLAST}}
BLAST is an automatic verification tool for checking the temporal safety properties of C programs. Given a C program and a temporal safety property, BLAST either statically proves the program satisfies the property or provides an execution path that exhibits a violation of the property \cite{blast}.
\par
Where BLAST differs from PAT, is that it no longer relies on process algebra. The model checker is capable of running directly on a subset of C programs where no intermediate modelling is required. As an end-user tool, this is more generally applicable than PAT; there is no burden on developers to think about how to model their systems with process algebra and instead can directly get safety guarantees from their programs. BLAST handles the translation of C programs to an abstract reachability tree (ART), a labeled tree that represents a portion of the reachable state space of the program. Using a context-free reachability algorithm on this representation of a C program, means temporal properties can be checked, without the end programmer being required to think about what the control-flow automata for the program will look like.
\par
BLAST falls short when model-checking large C programs. More importantly, it is unable to provide any guarantees on concurrent programs. We are primarily concerned with concurrent systems in this report, as Elixir is a language that is designed for building concurrent systems.

\subsubsection*{\textbf{PRISM}}
PRISM is a probabilistic model checker, a tool for formal modelling and analysis of systems that exhibit random behaviour or probabilistic behaviour \cite{prism}. It has been used to analyse systems implementing random distributed algorithms.

\subsubsection*{\textbf{TLC}}
In 1980, Leslie Lamport discovered the Temporal Logic of Action (TLA) \cite{tla}. TLA is a logic system for specifying and reasoning about concurrent systems. Both the systems and their properties are represented in the same logic, so that the assertion that a system meets its specification, can be expressed by a logical implication.
\par
TLA is capable of specifying complex systems but in a typically verbose manner. Leslie Lamport introduced TLA+ \cite{tlaplus}, combining mathematical ideas with concepts from programming languages to create a specification language that would allow mathematicians to write specifications in 20 lines as opposed to 20 pages.
\par
Furthering on from Leslie Lamport's discovery of these specification languages, Lamport created TLC \cite{tlc}, a model checker for the verification of TLA+ specifications. Similarly to BLAST, TLC builds a finite-state machine from the specification so the model checker can verify and debug invariance properties over it. TLC has been used to verify many large-scale, real-world systems specified in TLA+. Not only does it verify temporal properties of TLA+ specifications, but it can also model check PlusCal \cite{pluscal} algorithms. PlusCal is an algorithm language aimed to resemble that of pseudocode, but PlusCal algorithms can be automatically translated to TLA+ specifications to be reasoned about formally with TLC. We have already come across the concept of model-checking algorithms as opposed to specifications with BLAST, but instead of being strictly bound to the C programming language, PlusCal provides a more general framework agnostic of a choice of programming language allowing developers to separate reasoning about algorithms from their respective programs.

\subsubsection*{\textbf{SPIN}}
SPIN is an efficient verification system for models of distributed software systems. It has been used to detect design errors in applications ranging from high-level descriptions of distributed algorithms to detailed code \cite{spin}. Spin has a specification language, Process Meta Language (Promela), which the model checker uses to prove the correctness of asynchronous process interactions. Spin supports asynchronous process communication through channels, where processes can send and receive messages. Spin constructs labeled transition systems for respective processes from Promela specifications which it goes on to use for scheduling and to reason about properties of the model. Because many programming languages, such as GO \cite{go} rely on the creation of channels for asynchronous communication between processes, Promela becomes a natural solution to modelling these systems.
\subsubsection{Summary}
We have discussed a selection of model-checkers and what their primary focus is. Many existing model-checkers have been originally designed to prove specifications over sequential models. Some have taken this further and applied model checking directly over programming languages, such as BLAST. Other model-checkers have introduced some primitives for reasoning about concurrency. TLC allows for the specification of processes and using structures can begin to specify shared memory. Similarly, SPIN allows processes to be specified and supports the creation of channels for communication. Despite this, none of the model checkers discussed include message-passing as a first-class construct. To reason about message-passing models, such as the actor model, work has to be done to formalise actor-based constructs. This makes specifying actor-based systems, such as systems written in Elixir a non-trivial task.
\subsection{Additional Verification Techniques}
Alongside model checking, there are other techniques that can be used in system verification, which are worth briefly mentioning. 
\subsubsection{Theorem Proving}
Theorem proving is another process to verify programs. In theorem proving, axioms are applied to a set of statements to determine if a particular statement holds. For example, Z3 \cite{z3} is a satisfiability modulo theories (SMT) solver developed by Microsoft that can verify propositional logic assertions.
\subsubsection{Hoare Logic}
Hoare Logic was discovered in 1969 by Tony Hoare \cite{hoare_logic}. Hoare Logic defines the Hoare Triple, an essential idea in describing how code execution changes the state of a computation. A Hoare Triple is composed of a pre-condition assertion $P$, a post-condition assertion $Q$, and a command $C$.
\[
\{P\} C \{Q\}
\]
\par
Hoare Logic provides axioms and inference rules required to construct a simple imperative programming language. If $P$ holds in the given state and $C$ terminates, then $Q$ will hold after. Below is an example of a simple Hoare Triple for the \texttt{skip} command, which leaves the program state unchanged.
\[
    \{P\} \text{skip} \{P\}
\]
Note how the postcondition is the same as the precondition for this command.
\par
Hoare describes many more rules that allow for assignment, composition, consequence and so forth. These rules have led to the development of modern-day theorem provers, such as Z3, which will detailed more later.
\par
To help understand, we show a concrete example of a Hoare Triple. In this example, the pre-condition $P$ and post-condition $Q$ represent the known program state for a variable, $x$. We informally describe a command $C$ as an assignment to $x$ that modifies the known state of the program.
\[
\{ x \rightarrow 1 \} \text{ x := x + 1 } \{ x \rightarrow 2 \}
\]
To formalise this notation, we should define rules for commands, but for brevity, these have been omitted.

\section{Existing Work} \label{sec:existing_work}
Much work has gone into model checking, theorem-proving and verifying the implementations of systems. For Elixir, there are tools such as dialyzer \cite{dialyzer}, which statically analyse Elixir programs for type errors or dead code. Whilst tools like such provide Elixir developers better guarantees their code is correct, it does not verify the correctness of a system as a whole. Elixir also has libraries for property-based testing, such as PropEr \cite{proper}, which can be used to generate random test cases for a system. Property-based testing randomly generates inputs to test a system, which can be useful for finding edge cases that unit tests may not cover. However, property-based testing does not provide guarantees about the correctness of a system, instead, it is used to find bugs in code. Work has also gone into verifying message-passing in Elixir using binary session types \cite{binary_session_types}. This approach ensures two processes communicating over compatible protocols avoid certain communication errors (i.e. hanging messages), but has not been extended to multiparty session types, so is not appropriate for verifying all actor-based systems.
\par
There is also existing work in the greater verification of real-world software. Much of this is done on sequentially executed programs as concurrency introduces a new level of complexity. For example, both C programs and GO programs have previously been targetted as good options for model checking \cite{gomela, c_to_promela}. While these tools provide system guarantees, they primarily focus on detecting deadlocks or dataraces within a system and do not support other safety or liveness properties.  
\subsection{Verification-aware Languages}
Verification-aware languages are a new trend in programming languages, where the language is designed to support proving the correctness of a program. Examples of these include Lean, Dafny and Boogie. We will explore some of these languages in detail to understand how verification-aware languages can be a powerful tool to reason about the correctness of a system. The conventional alternatives involve either disregarding formal methods entirely or hand translating a program into a specification language, such as TLA+. A good verification-aware language should naturally integrate the system specification with the implementation. The aim is to reduce the burden on programmers to maintain seperate specifications alongside evolving codebases.
\subsubsection{Lean}
The Lean theorem prover is a proof assistant developed by Leonardo de Moura \cite{lean}. Lean is first and foremost a functional programming language designed to write correct and maintainable code. Lean can be used as an interactive theorem prover, where developers can write proofs alongside code. It supports many features of modern-day functional languages, such as first-class functions, pattern matching and even multithreading. A proof assistant is a language that allows developers to define objects and specifications over them. They can be used to verify the correctness of programs (similar to a model checker) as they check proofs are correct using logical foundations. The theorem proofs typically involve solving constraint problems, by determining if a first-order formula can be satisfied concerning constraints generated during analysis of functions.
\par
While lean is itself both a functional programming language and theorem prover, this approach differs in implementation from other theorem provers, such as Dafny, which instead prove theorems using existing backend theorem provers.

\subsubsection{Dafny}
Dafny is a verification-aware programming language that has native support for inlining specifications that can be verified by a theorem prover \cite{dafny_paper}. Dafny aims to modernise the approach developers take to designing systems, by encouraging developers to write correct specifications instead of necessarily correct code. With the rise of modern theorem provers, this untraditional approach is now realistic. Dafny is an imperative language with methods, variables, loops and many other features of typical imperative programming languages. Dafny programs are equipped with supporting tools to translate to other imperative languages, such as Java and Python. 
\par
Dafny verifies the correctness of programs using the theorem prover, Z3 \cite{z3}. Developers can write specifications alongside code, such as methods, which can then be directly verified. The format of specifications typically follows those of a Hoare Triple, $\{P\}C\{Q\}$, such that given a precondition, $\{P\}$ holds, if $C$ terminates, a postcondition, $\{Q\}$, will hold. In Dafny, the language reserves the keywords \texttt{requires} and \texttt{ensures} for pre and postconditions. Listing \ref{fig:dafny_add} shows a basic example of a Dafny method, which introduces an \texttt{Add} method. The implementation unintentionally introduces a bug such that, any execution paths with an input $\{ a \in \mathbb{Z} \mid a < 0 \}$ do not necessarily return the sum of the two inputs. Because Dafny places the burden on writing good specifications as opposed to correct code, the underlying theorem prover can use our postcondition to flag that this program is not correct for all execution paths.
\begin{lstlisting}[language=Dafny, xleftmargin=.3\linewidth, caption={Example of a method in Dafny}., label={fig:dafny_add}]
    method Add(a: int, b: int) returns (c: int)
        ensures c == a + b;
    {
        if  a < 0 {
            c := -1;
        } else {
            c := a + b;
        }
    }
\end{lstlisting}
\par
Listing \ref{fig:dafny_add} only gives a small insight into the power the Dafny specification language defines. Alongside the evaluation of basic expressions, Dafny allows the use of quantifiers such as the universal quantifier. The introduction of quantifiers allows us to write pre and postconditions over collections of objects, such as sets and arrays. Listing \ref{fig:dafny_quantifier} shows a basic example of how the universal quantifier can be used with the underlying theorem prover, to assert all the elements of an array, \texttt{a[]}, are strictly positive.
\begin{lstlisting}[language=Dafny, numbers=none, xleftmargin=.3\linewidth, caption={\texttt{forall} quantifier in Dafny \cite{dafny_tutorial}}., label={fig:dafny_quantifier}]
forall k: int :: 0 <= k < a.Length ==> 0 < a[k]
\end{lstlisting}
\par
Dafny also uses other concepts that support the verification of programs. Assertions can be used to provide guarantees in the middle of a method. Loop invariants can annotate while loops to check a condition holds upon entering a loop and after every execution of the loop body. Similarly, loop variants can be used to determine termination of while loops, by checking that every execution of a loop body makes progress towards the bound of the loop.
\subsubsection{Boogie}
Boogie is a modelling language intended as an intermediate verification language (IVL), developed at Microsoft \cite{boogie}. The language is described as an intermediate language because it is designed to bridge the gap between a program and a program verifier. Many tools that rely on Boogie's intermediate representation are doing so to translate source code in a native language into a format that can be proved. Dafny is a prime example of a programming language which does so. The Dafny compiler generates Boogie programs that can then be verified by Z3. This provides multiple benefits for Dafny. Firstly, Dafny does not have to concern itself with being dependent on a specific SMT solver, such as Z3, instead, it can be designed agnostic to the choice of theorem prover as Boogie will take responsibility for handling interaction with theorem provers. Boogie also bears a closer resemblance to an imperative programming language (like Dafny), so translation between the two is easier than translating to Z3. Listing \ref{fig:boogie_add} shows an example Boogie program, defining a single procedure, \texttt{add}, that represents the translated code from the Dafny example in listing \ref{fig:dafny_add}. Note the similarities between both programming languages, both use \texttt{ensures} to capture preconditions and have very similar syntax and control flow. However, now that our program is written in the Boogie IVL, we can directly determine an execution path that violates the precondition using a theorem prover such as Z3.
\begin{lstlisting}[language=boogie, xleftmargin=.3\linewidth, caption={An example Boogie IVL program}., label={fig:boogie_add}]
procedure add(a: int, b: int) returns (c: int)
    ensures c == a + b;
{
    if (a < 0) 
    {
        c := -1;
    } else {
        c := a + b;
    }
}
\end{lstlisting}

\subsection{Deadlock Detection}
Alongside research into verification-aware languages, there has been work into detecting deadlocks in concurrent systems. These approaches typically involve the use of model checkers to determine if a system can reach a deadlock state. For example, Java Pathfinder \cite{jpf} is a model checker for Java programs. It can be used to detect deadlocks and data races. The initial version of Java Pathfinder was a translator from Java to Promela. Since, Java Pathfinder now uses a Java Virtual Machine (JVM) implementation directly to model check Java programs.
\par
There has also been work into detecting deadlocks in GO programs. Gomela \cite{gomela} was proven to catch more deadlocks than GCatch \cite{gcatch} and Godel2 \cite{godel2}. Gomela focuses on channeled communication between goroutines. Similarly, deadlock detection has been researched for C programs \cite{c_to_promela}. 
\par
None of these approaches capture the semantics involved in a pure message-based, actor model. They also do not provide guarantees about the liveness properties of a system. This is a limitation in existing research of verification tools concerning modern programming languages.
\section{Summary}
This chapter has provided an overview of core concepts related to concurrent programs and verification of them. We saw process algebra that can be used to model and reason about concurrent processes, as well as Hoare Logic and its definition of the Hoare Triple as a fundamental property in verification. We also looked at applications based on this theory, such as model checkers, theorem provers and programming languages. Much work related to the topic of verifying programming languages was explored, but importantly, we learned about SPIN. We also learned about Boogie, the intermediate verification language that can verify programmatic assumptions using Z3. This chapter also discussed some of the limitations in existing research, such as a need for new techniques to verify the liveness properties of real-world systems. The next chapter will discuss the Elixir programming language.

