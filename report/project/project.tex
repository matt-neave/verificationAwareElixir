\chapter{Veriflixir}
Veriflixir is the main project contribution. The Veriflixir toolchain supports the simulation and verification of a set of Elixir programs. This set is named LTLixir and is detailed in section \ref{sec:ltlixir}. This chapter aims to inform the reader of the constructs defined in LTLixir and how Veriflixir can be used to reason about them. \ref{sec:ltlixir} introduces the LTLixir language and its constructs. \ref{sec:verifiable} provides an example of specifying a verifiable system and how Veriflixir can be used to detect violations of a specification. The subsequent subsections provide further details of more interesting features of LTLixir, such as specifying temporal properties.
\section{LTLixir} \label{sec:ltlixir}
LTLixir is the multi-purpose specification language that compiles to BEAM byte-code and is supported for verification by Veriflixir. Primarily, LTLixir is a subset of Elixir supporting both sequential and concurrent execution. This subset is expressive enough to well-known distributed algorithms such as basic paxos \cite{basic-paxos} and the alternating-bit protocol \cite{ab-protocol}. LTLixir extends Elixir with constructs for specifying temporal properties, specifically LTL properties (where LTLixir derives its name) as well as Floyd-Hoare style logic for specifying pre- and post-conditions. Specifications can be parameterized to identify violations of properties on specific configurations. 
\par

\section{Constructing a Verifiable Elixir Program} \label{sec:verifiable}
\subsection{Detecting a Deadlock} \label{sec:deadlock}
\subsection{Inspecting a Trace} 
\subsection{Linear Temporal Logic} 
\subsection{Hoare-style Logic} 